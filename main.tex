\documentclass[a4paper,10pt]{article}
\usepackage[T1]{fontenc}
\usepackage[utf8]{inputenc}
\usepackage{amssymb}
\usepackage[polish]{babel}
\usepackage{amsthm}
\usepackage{amsmath}
\usepackage{times}
\usepackage{anysize}
\usepackage{enumerate}
\usepackage{color}
\usepackage{tikz}

\marginsize{1cm}{1cm}{1cm}{1cm}
\definecolor{darkgray}{gray}{0.3}
\definecolor{darkblue}{RGB}{0,0,180}
\definecolor{red}{RGB}{255,0,0}
\definecolor{lightgray}{gray}{0.6}
\sloppy 

\begin{document}

Projekt domyślnie ma mieć Sourceforge'a (nie mam narazie czasu go ogarniać) i w połączeniu z Gitem by się go rozwijało.
Jeśli masz wolnego czasu troche i chcesz pomóc doślij jakieś rozwiązanie.
Bardzo potrzebne są korekty, napewno jest tu sporo błędów.
Potrzebni są też komentatorzy, chodzi o całkowicie łopatologiczny komentarz typu "tu liczymy delte bo..."\\
Jeśli jeszcze nie ma źródeł to niedługo będą na FTP/II rok/Systemy Dynamiczne. Ale dobrze by było jakbyście pisali że coś zmieniacie, bo dopuki nie ma tego na Gitcie możemy sobie nadpisywać i sporo tracić.\\
Z tego przedmiotu jest egzamin, więc warto to zrobić


\include{week3} % do tego pliku wrzucić tydzień po skończeniu prac nad nim

\pagebreak
\section*{Autorzy:}

\subsection*{Skład:} 
Jacek Pietras

\subsection*{Rozwiązania:} 
Magdalena Warzesia\\
Ania Szarawara\\
irytek102\\
Gniewomir\\
Jacek Pietras

\subsection*{Komentarze:} 
\end{document}


%  \left[ \begin{array}{cc}     &\\&    \end{array}\right]
%  \left[ \begin{array}{c}     \\    \end{array}\right]



