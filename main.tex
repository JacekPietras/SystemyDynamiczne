%%& -job-name="[SD] Zadania v0.11"
\documentclass[a4paper,10pt]{article}
\usepackage[T1]{fontenc}
\usepackage[utf8]{inputenc}
\usepackage{amssymb}
\usepackage[polish]{babel}
\usepackage{amsthm}
\usepackage{amsmath}
\usepackage{times}
\usepackage{anysize}
\usepackage{enumerate}
\usepackage{color}
\usepackage{circuitikz}

\usepackage{siunitx}
\usepackage{mathrsfs}
\usepackage{cancel}


\marginsize{1cm}{1cm}{1cm}{1cm}
\definecolor{darkgray}{gray}{0.3}
\definecolor{darkblue}{RGB}{0,0,180}
\definecolor{red}{RGB}{255,0,0}
\definecolor{lightgray}{gray}{0.6}
\sloppy 

\newcommand*\circled[1]{\tikz[baseline=(char.base)]{
            \node[shape=circle,draw,inner sep=1pt] (char) {\small{#1}};}}

\begin{document}
\noindent https://github.com/Jock69pl/SystemyDynamiczne.git\\
\\
Jeśli masz wolnego czasu troche i chcesz pomóc doślij jakieś rozwiązanie.\\
Bardzo potrzebne są korekty, napewno jest tu sporo błędów.\\
Potrzebni są też komentatorzy, chodzi o całkowicie łopatologiczny komentarz typu "tu liczymy delte bo..."\\
\\
Z tego przedmiotu jest egzamin, więc warto to zrobić


\pagebreak
\section*{Tydzień 3}
Dyskretne systemy dynamiczne
%###################                 3.1.1              #################################%
\subsection*{Zadanie 3.1.1} {\color{darkgray}
	Narysować rozwiązanie równania różnicowego $x(k+1)=\lambda_ix(k)$\\
	 dla $i=1,2,3$ gdzie \\
	$\lambda_1 = 0, \lambda_2=1, \lambda_3=-1$ \\
	przy czym $x(0)=-1$ i $k\geq0$\\
}\lineh
\\\\
\noindent$\bullet\ \ i=3, \lambda_i = -1$\\
$x(k)=(-1)^k(-1)=(-1)^{k+1}$

\begin{figure}[!h]
\begin{tikzpicture}[scale=0.6]
	\draw[thick][->](0,0)--(6,0) node [right=3pt]{$k$};
	\draw[thick][->](0,-3)--(0,3) node [right=3pt]{$x(k)$};

	\foreach \y [count=\i] in {-2,...,-1} {{
	     \draw (-0.1,\y) -- (0.1,\y) node [left=2pt]{{\tiny \y}};
 	}}
	\foreach \y [count=\i] in {1,...,2} {{
	     \draw (-0.1,\y) -- (0.1,\y) node [left=2pt]{{\tiny \y}};
 	}}
	\foreach \x [count=\i] in {-4,...,-1} {{
	    % \draw (\x,-0.1) -- (\x,0.1) node [below=4pt]{{\tiny \x}};
 	}}
	\foreach \x [count=\i] in {1,...,5} {{
	     \draw (\x,-0.1) -- (\x,0.1) node [below=4pt]{{\tiny \x}};
 	}}
		
	% koniec Osi
	
	\fill[fill=blue](0,-1) circle(4pt);
	\fill[fill=blue](1,1) circle(4pt);
	\fill[fill=blue](2,-1) circle(4pt);
	\fill[fill=blue](3,1) circle(4pt);
	\fill[fill=blue](4,-1) circle(4pt);
	\fill[fill=blue](5,1) circle(4pt);
	\fill[fill=blue](6,-1) circle(4pt);
\end{tikzpicture}
\end{figure}


\noindent$\bullet\ \ i=2, \lambda_i = 1$\\
$x(k)=(1)^k(-1)=-1$

\begin{figure}[!h]
\begin{tikzpicture}[scale=0.6]
	\draw[thick][->](0,0)--(6,0) node [right=3pt]{$k$};
	\draw[thick][->](0,-3)--(0,3) node [right=3pt]{$x(k)$};

	\foreach \y [count=\i] in {-2,...,-1} {{
	     \draw (-0.1,\y) -- (0.1,\y) node [left=2pt]{{\tiny \y}};
 	}}
	\foreach \y [count=\i] in {1,...,2} {{
	     \draw (-0.1,\y) -- (0.1,\y) node [left=2pt]{{\tiny \y}};
 	}}
	\foreach \x [count=\i] in {-4,...,-1} {{
	    % \draw (\x,-0.1) -- (\x,0.1) node [below=4pt]{{\tiny \x}};
 	}}
	\foreach \x [count=\i] in {1,...,5} {{
	     \draw (\x,-0.1) -- (\x,0.1) node [below=4pt]{{\tiny \x}};
 	}}
		
	% koniec Osi
	
	\fill[fill=blue](0,-1) circle(4pt);
	\fill[fill=blue](1,-1) circle(4pt);
	\fill[fill=blue](2,-1) circle(4pt);
	\fill[fill=blue](3,-1) circle(4pt);
	\fill[fill=blue](4,-1) circle(4pt);
	\fill[fill=blue](5,-1) circle(4pt);
	\fill[fill=blue](6,-1) circle(4pt);
\end{tikzpicture}
\end{figure}

\noindent$\bullet\ \ i=0, \lambda_i = 0$\\
$x(k)=(0)^k(-1)=
	\begin{cases}
   	-1 &\text{dla } k=0\\
	0  &\text{dla } k={1,2,...}
	\end{cases}$

\begin{figure}[!h]
\begin{tikzpicture}[scale=0.6]
	\draw[thick][->](0,0)--(6,0) node [right=3pt]{$k$};
	\draw[thick][->](0,-3)--(0,3) node [right=3pt]{$x(k)$};

	\foreach \y [count=\i] in {-2,...,-1} {{
	     \draw (-0.1,\y) -- (0.1,\y) node [left=2pt]{{\tiny \y}};
 	}}
	\foreach \y [count=\i] in {1,...,2} {{
	     \draw (-0.1,\y) -- (0.1,\y) node [left=2pt]{{\tiny \y}};
 	}}
	\foreach \x [count=\i] in {-4,...,-1} {{
	    % \draw (\x,-0.1) -- (\x,0.1) node [below=4pt]{{\tiny \x}};
 	}}
	\foreach \x [count=\i] in {1,...,5} {{
	     \draw (\x,-0.1) -- (\x,0.1) node [below=4pt]{{\tiny \x}};
 	}}
		
	% koniec Osi
	
	\fill[fill=blue](0,-1) circle(4pt);
	\fill[fill=blue](1,0) circle(4pt);
	\fill[fill=blue](2,0) circle(4pt);
	\fill[fill=blue](3,0) circle(4pt);
	\fill[fill=blue](4,0) circle(4pt);
	\fill[fill=blue](5,0) circle(4pt);
	\fill[fill=blue](6,0) circle(4pt);
\end{tikzpicture}
\end{figure}

\pagebreak
%###################                 3.1.2              #################################%
\subsection*{Zadanie 3.1.2} {\color{darkgray}
	Narysować rozwiązanie równania różnicowego $x(k+1)=\lambda_ix(k)$\\
	 dla $i=1,2,3$ gdzie \\
	$\lambda_1 = -1, \lambda_2=-\frac{1}{2}, \lambda_3=1$ \\
	przy czym $x(0)=1$ i $k\geq0$\\
}\lineh
\\\\
\noindent$\bullet\ \ \lambda_1 = -1$\\
$x(k+1)=-x(k)$\\
$x(k)=(-1)^k\cdot1=(-1)^k$

\begin{figure}[!h]
\begin{tikzpicture}[scale=0.6]
	\draw[thick][->](0,0)--(6,0) node [right=3pt]{$k$};
	\draw[thick][->](0,-3)--(0,3) node [right=3pt]{$x(k)$};

	\foreach \y [count=\i] in {-2,...,-1} {{
	     \draw (-0.1,\y) -- (0.1,\y) node [left=2pt]{{\tiny \y}};
 	}}
	\foreach \y [count=\i] in {1,...,2} {{
	     \draw (-0.1,\y) -- (0.1,\y) node [left=2pt]{{\tiny \y}};
 	}}
	\foreach \x [count=\i] in {-4,...,-1} {{
	    % \draw (\x,-0.1) -- (\x,0.1) node [below=4pt]{{\tiny \x}};
 	}}
	\foreach \x [count=\i] in {1,...,5} {{
	     \draw (\x,-0.1) -- (\x,0.1) node [below=4pt]{{\tiny \x}};
 	}}
		
	% koniec Osi
	
	\fill[fill=blue](0,1) circle(4pt);
	\fill[fill=blue](1,-1) circle(4pt);
	\fill[fill=blue](2,1) circle(4pt);
	\fill[fill=blue](3,-1) circle(4pt);
	\fill[fill=blue](4,1) circle(4pt);
	\fill[fill=blue](5,-1) circle(4pt);
	\fill[fill=blue](6,1) circle(4pt);
\end{tikzpicture}
\end{figure}


\noindent$\bullet\ \ \lambda_2 = -\frac{1}{2}$\\
$x(k+1)=-\frac{1}{2}\cdot x(k)$\\
$x(k)=(-\frac{1}{2})^k\cdot 1=(-\frac{1}{2})^k$

\begin{figure}[!h]
\begin{tikzpicture}[scale=0.6]
	\draw[thick][->](0,0)--(6,0) node [right=3pt]{$k$};
	\draw[thick][->](0,-3)--(0,3) node [right=3pt]{$x(k)$};

	\foreach \y [count=\i] in {-2,...,-1} {{
	     \draw (-0.1,\y) -- (0.1,\y) node [left=2pt]{{\tiny \y}};
 	}}
	\foreach \y [count=\i] in {1,...,2} {{
	     \draw (-0.1,\y) -- (0.1,\y) node [left=2pt]{{\tiny \y}};
 	}}
	\foreach \x [count=\i] in {-4,...,-1} {{
	    % \draw (\x,-0.1) -- (\x,0.1) node [below=4pt]{{\tiny \x}};
 	}}
	\foreach \x [count=\i] in {1,...,5} {{
	     \draw (\x,-0.1) -- (\x,0.1) node [below=4pt]{{\tiny \x}};
 	}}
		
	% koniec Osi
	
	\fill[fill=blue](0,1) circle(4pt);
	\fill[fill=blue](1,-0.5) circle(4pt);
	\fill[fill=blue](2,0.25) circle(4pt);
	\fill[fill=blue](3,-0.125) circle(4pt);
	\fill[fill=blue](4,0.0625) circle(4pt);
\end{tikzpicture}
\end{figure}

\noindent$\bullet\ \ \lambda_3 = 1$\\
$x(k+1)=x(k)$\\
$x(k)=1^k\cdot1 =1$

\begin{figure}[!h]
\begin{tikzpicture}[scale=0.6]
	\draw[thick][->](0,0)--(6,0) node [right=3pt]{$k$};
	\draw[thick][->](0,-3)--(0,3) node [right=3pt]{$x(k)$};

	\foreach \y [count=\i] in {-2,...,-1} {{
	     \draw (-0.1,\y) -- (0.1,\y) node [left=2pt]{{\tiny \y}};
 	}}
	\foreach \y [count=\i] in {1,...,2} {{
	     \draw (-0.1,\y) -- (0.1,\y) node [left=2pt]{{\tiny \y}};
 	}}
	\foreach \x [count=\i] in {-4,...,-1} {{
	    % \draw (\x,-0.1) -- (\x,0.1) node [below=4pt]{{\tiny \x}};
 	}}
	\foreach \x [count=\i] in {1,...,5} {{
	     \draw (\x,-0.1) -- (\x,0.1) node [below=4pt]{{\tiny \x}};
 	}}
		
	% koniec Osi
	
	\fill[fill=blue](0,1) circle(4pt);
	\fill[fill=blue](1,1) circle(4pt);
	\fill[fill=blue](2,1) circle(4pt);
	\fill[fill=blue](3,1) circle(4pt);
	\fill[fill=blue](4,1) circle(4pt);
	\fill[fill=blue](5,1) circle(4pt);
	\fill[fill=blue](6,1) circle(4pt);
\end{tikzpicture}
\end{figure}

\pagebreak
%###################                 3.1.3              #################################%
\subsection*{Zadanie 3.1.3} {\color{darkgray}
	Narysować rozwiązanie równania różnicowego $x(k+1)=\lambda_ix(k)$\\
	 dla $i=1,2,3$ gdzie \\
	$\lambda_1 = 0, \lambda_2=1, \lambda_3=-1$ \\
	przy czym $x(0)=1$ i $k\geq0$\\
}\lineh
\\\\
\noindent$\bullet\ \lambda_1 = 0$\\

\begin{figure}[!h]
\begin{tikzpicture}[scale=0.6]
	\draw[thick][->](0,0)--(6,0) node [right=3pt]{$k$};
	\draw[thick][->](0,-3)--(0,3) node [right=3pt]{$x(k)$};

	\foreach \y [count=\i] in {-2,...,-1} {{
	     \draw (-0.1,\y) -- (0.1,\y) node [left=2pt]{{\tiny \y}};
 	}}
	\foreach \y [count=\i] in {1,...,2} {{
	     \draw (-0.1,\y) -- (0.1,\y) node [left=2pt]{{\tiny \y}};
 	}}
	\foreach \x [count=\i] in {-4,...,-1} {{
	    % \draw (\x,-0.1) -- (\x,0.1) node [below=4pt]{{\tiny \x}};
 	}}
	\foreach \x [count=\i] in {1,...,5} {{
	     \draw (\x,-0.1) -- (\x,0.1) node [below=4pt]{{\tiny \x}};
 	}}
		
	% koniec Osi
	
	\fill[fill=blue](0,1) circle(4pt);
	\fill[fill=blue](1,0) circle(4pt);
	\fill[fill=blue](2,0) circle(4pt);
	\fill[fill=blue](3,0) circle(4pt);
	\fill[fill=blue](4,0) circle(4pt);
	\fill[fill=blue](5,0) circle(4pt);
	\fill[fill=blue](6,0) circle(4pt);
\end{tikzpicture}
\end{figure}


\noindent$\bullet\ \lambda_2 = 1$\\

\begin{figure}[!h]
\begin{tikzpicture}[scale=0.6]
	\draw[thick][->](0,0)--(6,0) node [right=3pt]{$k$};
	\draw[thick][->](0,-3)--(0,3) node [right=3pt]{$x(k)$};

	\foreach \y [count=\i] in {-2,...,-1} {{
	     \draw (-0.1,\y) -- (0.1,\y) node [left=2pt]{{\tiny \y}};
 	}}
	\foreach \y [count=\i] in {1,...,2} {{
	     \draw (-0.1,\y) -- (0.1,\y) node [left=2pt]{{\tiny \y}};
 	}}
	\foreach \x [count=\i] in {-4,...,-1} {{
	    % \draw (\x,-0.1) -- (\x,0.1) node [below=4pt]{{\tiny \x}};
 	}}
	\foreach \x [count=\i] in {1,...,5} {{
	     \draw (\x,-0.1) -- (\x,0.1) node [below=4pt]{{\tiny \x}};
 	}}
		
	% koniec Osi
	
	\fill[fill=blue](0,1) circle(4pt);
	\fill[fill=blue](1,1) circle(4pt);
	\fill[fill=blue](2,1) circle(4pt);
	\fill[fill=blue](3,1) circle(4pt);
	\fill[fill=blue](4,1) circle(4pt);
	\fill[fill=blue](5,1) circle(4pt);
	\fill[fill=blue](6,1) circle(4pt);
\end{tikzpicture}
\end{figure}

\noindent$\bullet\ \lambda_3 = -1$\\

\begin{figure}[!h]
\begin{tikzpicture}[scale=0.6]
	\draw[thick][->](0,0)--(6,0) node [right=3pt]{$k$};
	\draw[thick][->](0,-3)--(0,3) node [right=3pt]{$x(k)$};

	\foreach \y [count=\i] in {-2,...,-1} {{
	     \draw (-0.1,\y) -- (0.1,\y) node [left=2pt]{{\tiny \y}};
 	}}
	\foreach \y [count=\i] in {1,...,2} {{
	     \draw (-0.1,\y) -- (0.1,\y) node [left=2pt]{{\tiny \y}};
 	}}
	\foreach \x [count=\i] in {-4,...,-1} {{
	    % \draw (\x,-0.1) -- (\x,0.1) node [below=4pt]{{\tiny \x}};
 	}}
	\foreach \x [count=\i] in {1,...,5} {{
	     \draw (\x,-0.1) -- (\x,0.1) node [below=4pt]{{\tiny \x}};
 	}}
		
	% koniec Osi

	\fill[fill=blue](0,1) circle(4pt);
	\fill[fill=blue](1,-1) circle(4pt);
	\fill[fill=blue](2,1) circle(4pt);
	\fill[fill=blue](3,-1) circle(4pt);
	\fill[fill=blue](4,1) circle(4pt);
	\fill[fill=blue](5,-1) circle(4pt);
	\fill[fill=blue](6,1) circle(4pt);
	
\end{tikzpicture}
\end{figure}
























\pagebreak
%###################                 3.2.1                #################################%
\subsection*{Zadanie 3.2.1} {\color{darkgray}
	Narysować rozwiązanie równania różnicowego $x(k+1)=x(k)+u_i(k)$\\
	dla $i=1,2,3$ gdzie \\
	$u_1 = -2, u_2=2,u_3=1$ \\
	przy czym $x(0)=0$ i $k\geq0$\\
}\lineh
\\\\
$x(k+1)=Ax(i)+Bu(i)$\\
$x(k)=A^kx(0)+\sum^{k-1}_{j=0}A^{k-1-j}Bu(j)$\\
${\color{lightgray}A=1, B=1, \forall j \ \ \ u(j)=u_i }$\\
$\Rightarrow x(k)=x(0)+\sum^{k-1}_{j=0}u$ dla $x(0)=0$\\
$\Rightarrow x(k)=u_i\cdot k$\\

\noindent$\bullet\ \ i=1, u_i = -2$

\begin{figure}[!h]
\begin{tikzpicture}[scale=0.5]
	\draw[thick][->](0,0)--(6,0) node [right=3pt]{$k$};
	\draw[thick][->](0,-6)--(0,2) node [right=3pt]{$x(k)$};

	\foreach \y [count=\i] in {-5,...,-1} {{
	     \draw (-0.1,\y) -- (0.1,\y) node [left=2pt]{{\tiny \y}};
 	}}
	\foreach \y [count=\i] in {1,...,1} {{
	     \draw (-0.1,\y) -- (0.1,\y) node [left=2pt]{{\tiny \y}};
 	}}
	\foreach \x [count=\i] in {-4,...,-1} {{
	    % \draw (\x,-0.1) -- (\x,0.1) node [below=4pt]{{\tiny \x}};
 	}}
	\foreach \x [count=\i] in {1,...,5} {{
	     \draw (\x,-0.1) -- (\x,0.1) node [below=4pt]{{\tiny \x}};
 	}}
		
	% koniec Osi
	
	\fill[fill=blue](0,0) circle(4pt);
	\fill[fill=blue](1,-2) circle(4pt);
	\fill[fill=blue](2,-4) circle(4pt);
	\fill[fill=blue](3,-6) circle(4pt);
\end{tikzpicture}
\end{figure}

\noindent$\bullet\ \ i=2, u_i = 2$

\begin{figure}[!h]
\begin{tikzpicture}[scale=0.5]
	\draw[thick][->](0,0)--(6,0) node [right=3pt]{$k$};
	\draw[thick][->](0,-2)--(0,6) node [right=3pt]{$x(k)$};

	\foreach \y [count=\i] in {-1,...,-1} {{
	     \draw (-0.1,\y) -- (0.1,\y) node [left=2pt]{{\tiny \y}};
 	}}
	\foreach \y [count=\i] in {1,...,5} {{
	     \draw (-0.1,\y) -- (0.1,\y) node [left=2pt]{{\tiny \y}};
 	}}
	\foreach \x [count=\i] in {-4,...,-1} {{
	    % \draw (\x,-0.1) -- (\x,0.1) node [below=4pt]{{\tiny \x}};
 	}}
	\foreach \x [count=\i] in {1,...,5} {{
	     \draw (\x,-0.1) -- (\x,0.1) node [below=4pt]{{\tiny \x}};
 	}}
		
	% koniec Osi
	
	\fill[fill=blue](0,0) circle(4pt);
	\fill[fill=blue](1,2) circle(4pt);
	\fill[fill=blue](2,4) circle(4pt);
	\fill[fill=blue](3,6) circle(4pt);
\end{tikzpicture}
\end{figure}

\noindent$\bullet\ \  i=3, u_i = 1$

\begin{figure}[!h]
\begin{tikzpicture}[scale=0.5]
	\draw[thick][->](0,0)--(6,0) node [right=3pt]{$k$};
	\draw[thick][->](0,-2)--(0,6) node [right=3pt]{$x(k)$};

	\foreach \y [count=\i] in {-1,...,-1} {{
	     \draw (-0.1,\y) -- (0.1,\y) node [left=2pt]{{\tiny \y}};
 	}}
	\foreach \y [count=\i] in {1,...,5} {{
	     \draw (-0.1,\y) -- (0.1,\y) node [left=2pt]{{\tiny \y}};
 	}}
	\foreach \x [count=\i] in {-4,...,-1} {{
	    % \draw (\x,-0.1) -- (\x,0.1) node [below=4pt]{{\tiny \x}};
 	}}
	\foreach \x [count=\i] in {1,...,5} {{
	     \draw (\x,-0.1) -- (\x,0.1) node [below=4pt]{{\tiny \x}};
 	}}
		
	% koniec Osi
	
	\fill[fill=blue](0,0) circle(4pt);
	\fill[fill=blue](1,1) circle(4pt);
	\fill[fill=blue](2,2) circle(4pt);
	\fill[fill=blue](3,3) circle(4pt);
	\fill[fill=blue](4,4) circle(4pt);
	\fill[fill=blue](5,5) circle(4pt);
\end{tikzpicture}
\end{figure}

\pagebreak
%###################                 3.2.2                #################################%
\subsection*{Zadanie 3.2.2} {\color{darkgray}
	Narysować rozwiązanie równania różnicowego $x(k+1)=2x(k)+u_i(k)$\\
	dla $i=1,2,3$ gdzie \\
	$u_1 = 0, u_2=-2,u_3=2$ \\
	przy czym $x(0)=0$ i $k\geq0$\\
}\lineh
\\\\
\noindent$\bullet\ \ u_1 = 0$\\
$x(k+1)=2x(k)+0$\\
$x(k)=2^k\cdot 0 = 0$

\begin{figure}[!h]
\begin{tikzpicture}[scale=0.5]
	\draw[thick][->](0,0)--(6,0) node [right=3pt]{$k$};
	\draw[thick][->](0,-2)--(0,2) node [right=3pt]{$x(k)$};

	\foreach \y [count=\i] in {-1,...,-1} {{
	     \draw (-0.1,\y) -- (0.1,\y) node [left=2pt]{{\tiny \y}};
 	}}
	\foreach \y [count=\i] in {1,...,1} {{
	     \draw (-0.1,\y) -- (0.1,\y) node [left=2pt]{{\tiny \y}};
 	}}
	\foreach \x [count=\i] in {-4,...,-1} {{
	    % \draw (\x,-0.1) -- (\x,0.1) node [below=4pt]{{\tiny \x}};
 	}}
	\foreach \x [count=\i] in {1,...,5} {{
	     \draw (\x,-0.1) -- (\x,0.1) node [below=4pt]{{\tiny \x}};
 	}}
		
	% koniec Osi
	
	\fill[fill=blue](0,0) circle(4pt);
	\fill[fill=blue](1,0) circle(4pt);
	\fill[fill=blue](2,0) circle(4pt);
	\fill[fill=blue](3,0) circle(4pt);
	\fill[fill=blue](4,0) circle(4pt);
	\fill[fill=blue](5,0) circle(4pt);
\end{tikzpicture}
\end{figure}

\noindent$\bullet\ \ u_2=-2$\\
$x(k+1)=2x(k)-2$\\
$x(k)=2^k \cdot 0 + \sum^{k-1}_{j=0}2^{k-1-j}\cdot(-2)=-2\cdot \sum^{k-1}_{j=0}(2^k\cdot 2^{-1}\cdot 2^{-j})$\\
$=-2^k\cdot \sum^{k-1}_{j=0}(\frac{1}{2})^j=-2^k\cdot\frac{1-(\frac{1}{2})^k}{1-\frac{1}{2}}=-2^{k+1}+2$

\begin{figure}[!h]
\begin{tikzpicture}[scale=0.5]
	\draw[thick][->](0,0)--(6,0) node [right=3pt]{$k$};
	\draw[thick][->](0,-7)--(0,2) node [right=3pt]{$x(k)$};

	\foreach \y [count=\i] in {-6,...,-1} {{
	     \draw (-0.1,\y) -- (0.1,\y) node [left=2pt]{{\tiny \y}};
 	}}
	\foreach \y [count=\i] in {1,...,1} {{
	     \draw (-0.1,\y) -- (0.1,\y) node [left=2pt]{{\tiny \y}};
 	}}
	\foreach \x [count=\i] in {-4,...,-1} {{
	    % \draw (\x,-0.1) -- (\x,0.1) node [below=4pt]{{\tiny \x}};
 	}}
	\foreach \x [count=\i] in {1,...,5} {{
	     \draw (\x,-0.1) -- (\x,0.1) node [below=4pt]{{\tiny \x}};
 	}}
		
	% koniec Osi
	
	\fill[fill=blue](0,0) circle(4pt);
	\fill[fill=blue](1,-2) circle(4pt);
	\fill[fill=blue](2,-6) circle(4pt);
\end{tikzpicture}
\end{figure}

\noindent$\bullet\ \ u_3=2$\\
$x(k+1)=2x(k)+2$\\
$x(k)=2^k \cdot 0 + \sum^{k-1}_{j=0}2^{k-1-j}\cdot(2)=2\cdot \sum^{k-1}_{j=0}(2^k\cdot 2^{-1}\cdot 2^{-j})$\\
$=2^k\cdot \sum^{k-1}_{j=0}(\frac{1}{2})^j=2^k\cdot\frac{1-(\frac{1}{2})^k}{1-\frac{1}{2}}=2^{k+1}-2$

\begin{figure}[!h]
\begin{tikzpicture}[scale=0.5]
	\draw[thick][->](0,0)--(6,0) node [right=3pt]{$k$};
	\draw[thick][->](0,-2)--(0,7) node [right=3pt]{$x(k)$};

	\foreach \y [count=\i] in {-1,...,-1} {{
	     \draw (-0.1,\y) -- (0.1,\y) node [left=2pt]{{\tiny \y}};
 	}}
	\foreach \y [count=\i] in {1,...,6} {{
	     \draw (-0.1,\y) -- (0.1,\y) node [left=2pt]{{\tiny \y}};
 	}}
	\foreach \x [count=\i] in {-4,...,-1} {{
	    % \draw (\x,-0.1) -- (\x,0.1) node [below=4pt]{{\tiny \x}};
 	}}
	\foreach \x [count=\i] in {1,...,5} {{
	     \draw (\x,-0.1) -- (\x,0.1) node [below=4pt]{{\tiny \x}};
 	}}
		
	% koniec Osi
	
	\fill[fill=blue](0,0) circle(4pt);
	\fill[fill=blue](1,2) circle(4pt);
	\fill[fill=blue](2,6) circle(4pt);
\end{tikzpicture}
\end{figure}




















\pagebreak
%###################                 3.3.1               #################################%
\subsection*{Zadanie 3.3.1} {\color{darkgray}
	Niech będzie dany układ ciągły opisany równaniem $\dot{x}(t)=ax(t)+bu(t)$\\
	(a) Znaleźć rozwiązanie tego układu dla: $x(0)=0, u(t)=1, a=-4, b=3$.\\
	(b) Znaleźć parametry układu dyskretnego, jeśli podł¡czono ekstrapolator pierwszego rzędu na wejściu
	i impulsator na wyjściu, przy czym pracuj¡ one synchronicznie z okresem próbkowania $t = 1$. 
	Rozwiązać powstały układ.\\
	(c) Jak wyżej, przy założeniu $t = 12$.\\
	(d) Narysować wszystkie rozwiązania na jednym układzie współrzędnych. Opisać różnice.\\
}\lineh
\\\\
Brak rozwiązania


\pagebreak
%###################                 3.3.2               #################################%
\subsection*{Zadanie 3.3.2} {\color{darkgray}
	Niech będzie dany układ ciągły opisany równaniem $\dot{x}(t)=ax(t)+bu(t)$\\
	(a) Znaleźć rozwiązanie tego układu dla: $x(0)=0, u(t)=1, a=-1, b=2$.\\
	(b) Znaleźć parametry układu dyskretnego, jeśli podłączono ekstrapolator pierwszego rzędu na wejściu
	i impulsator na wyjściu, przy czym pracują one synchronicznie z okresem próbkowania $t = 1$. 
	Rozwiązać powstały układ.\\
	(c) Jak wyżej, przy założeniu $t = 10$.\\
	(d) Narysować wszystkie rozwiązania na jednym układzie współrzędnych. Opisać różnice.\\
}\lineh
\\\\
(a)\\
$\dot{x}=-x+2$\\
$\frac{dx}{dt}=-x+2$\\
$\frac{dx}{-x+2}=dt$\\
$-\ln |-x+2|=t+c$, gdzie $c$ jest stałą \\
$\ln |-x+2|=-t+c$\\
$e^{-t+c}=-x+2$\\
$e^{-t}e^c=-x+2$\\
$ce^{-t}=-x+2$, bo $e^c$ to też stała\\

$x(0)=c+2=0 \Rightarrow c=-2$\\
$ce^{-t}=-x+2 \Rightarrow \boxed{x(t)=ce^{-t}+2=-2e^{-t}+2}$\\


\noindent
(b)\\
$h=1$\\
$x^+(i)=x(ih)=x(i)$\\
$u^+(i)=u(i)$\\
$y^+(i)=y(i)$\\
$A^+=e^{hA}=e^{-1}$\\
$B^+=\int_0^1e^{tA}B\ dt =\int_0^1e^{-t}2\ dt=2\int_0^1e^{-t}\ dt = -2e^t|^1_0=-2(e^{-1}-1)=2-\frac{2}{e}$\\
$C^+=C$\\
$x(i+1)=e^{-1}x(i)+(2-\frac{2}{e})u(i)=\frac{x(i)}{e}+2-\frac{2}{e}$\\
$x(k)=(e^{-1})^kx(0)+\sum_{j=0}^{k-1}(e^{-1})^{k-1-j}(2-\frac{2}{e})=(2-\frac{2}{e})\sum_{j=0}^{k-1}(e^{-k}\cdot e^1\cdot e^j)$\\
$=(2-\frac{2}{e})e^{1-k}\cdot\frac{1-e^k}{1-e}=2(e\cdot e^{-k} -e^{-k})\frac{1-e^k}{1-e}=2(-e^{-k})(1-e^k)=2-2e^{-k}$\\

\noindent
(c)\\
$h=10$\\
$x^+(i)=x(10i)$\\
$u^+(i)=u(10i)$\\
$A^+=e^{10A}=e^{-10}$\\
$B^+=2\int_0^{10}e^{-t} \ dt=2(-e^{-t}|_0^{10})=-2(e^{-10}-1)=2-2e^{-10}$\\
$x(k)=\underbrace{(e^{-10})^kx(0)}_{=0}+\sum_{j=0}^{k-1}(e^{-10})^{k-1-j}(2-2e^{-10})=(2-2e^{-10})\cdot e^{-10k-10}\cdot\frac{1-(e^{10})^k}{1-e^{10}}$\\
$=2(e^{-10k}\cdot e^{10}-e^{-10}\cdot e^{-10k}\cdot e^{10})\cdot\frac{1-e^{10k}}{1-e^{10}}=2(-e^{-10k}(1-e^{10}))\frac{1-e^{10k}}{1-e^{10}}=2-2e^{-10k}$\\

\begin{figure}[!h]
\begin{tikzpicture}[scale=0.5]
	\draw[draw=lightgray] (0,2) -- (10,2);
	\draw[thick][->](0,0)--(10,0) node [right=3pt]{$t$};
	\draw[thick][->](0,-2)--(0,3) node [right=3pt]{$x(t)$};

	\foreach \y [count=\i] in {-1,...,-1} {{
	     \draw (-0.1,\y) -- (0.1,\y) node [left=2pt]{{\tiny \y}};
 	}}
	\foreach \y [count=\i] in {1,...,2} {{
	     \draw (-0.1,\y) -- (0.1,\y) node [left=2pt]{{\tiny \y}};
 	}}
	\foreach \x [count=\i] in {1,...,9} {{
	     \draw (\x,-0.1) -- (\x,0.1) node [below=4pt]{{\tiny \x}};
 	}}
		
	% koniec Osi
	\draw (0,0)--(1,1.264)--(2,1.729)--(3,1.900)--(4,1.963)--(5,1.986)--(6,1.995)--(7,2)--(8,2) --(9,2);


	\fill[fill=blue](0,0) circle(4pt);
	\fill[fill=blue](1,1.264) circle(4pt);
	\fill[fill=blue](2,1.729) circle(4pt);
	\fill[fill=blue](3,1.900) circle(4pt);
	\fill[fill=blue](4,1.963) circle(4pt);
	\fill[fill=blue](5,1.986) circle(4pt);
	\fill[fill=blue](6,1.995) circle(4pt);
	\fill[fill=blue](7,2) circle(4pt);
	\fill[fill=blue](8,2) circle(4pt);
	\fill[fill=blue](9,2) circle(4pt);
\end{tikzpicture}
\end{figure}
- wykres (a) jest liniowy, (b) i (c) punktowe
- (a) punkty co 1, (b) punkty co 10 (?)




















\pagebreak
%###################                 3.4.1               #################################%
\subsection*{Zadanie 3.4.1} {\color{darkgray}
	Obliczyć $A^n$ dla macierzy\\
	$A=\left[ \begin{array}{cc} 4&-2\\0&2\end{array}\right]$\\
}\lineh
\\\\
$A^n=PJP^{-1}$\\
$A^n=\underbrace{(PJP^{-1})(PJP^{-1})...(PJP^{-1})}_{n}$\\
$A^n=PJ^nP^{-1}$\\
$| [A-\lambda I] |=0$\\
$(4-\lambda)(2-\lambda)=0 \Rightarrow \lambda_1=4, \lambda_2=2$\\
$[A-\lambda I]\omega_1=0$\\
\\
$\left[ \begin{array}{cc}     0&-2\\0&-2    \end{array}\right]\left[ \begin{array}{c}     \omega_{11}\\\omega_{12}    \end{array}\right]=0
\ \ \ \ \ \begin{array}{c}    -2\omega_{12}=0 \\\omega_{11} \in \mathbb{R}    \end{array}
\ \ \ \ \ \omega_1=\left[ \begin{array}{c}     \omega_{11}\\0    \end{array}\right]=\left[ \begin{array}{c}     1\\0    \end{array}\right]$\\
\\\\
$\left[ \begin{array}{cc}     2&-2\\0&0    \end{array}\right]\left[ \begin{array}{c}     \omega_{21}\\\omega_{22}    \end{array}\right]=0
\ \ \ \ \ \begin{array}{c}    2\omega_{21}-2\omega_{22}=0 \\ \omega_{22}=\omega_{21}\\ \omega_{21} \in \mathbb{R}    \end{array}
\ \ \ \ \ \omega_2=\left[ \begin{array}{c}     \omega_{21}\\\omega_{21}    \end{array}\right]=\left[ \begin{array}{c}     1\\1 \end{array}\right]$\\
$P=[\omega_{1} \ \  \omega_{2}]=\left[ \begin{array}{cc}    1 &1\\0& 1   \end{array}\right]$\\
$P^{-1}=\frac{1}{det(P)}\left[ \begin{array}{cc}    1 &1\\0& 1   \end{array}\right]=\left[ \begin{array}{cc}    1 &-1\\0& 1   \end{array}\right]$\\
$J=\left[ \begin{array}{cc}     4&0\\0&2    \end{array}\right],J^n=\left[ \begin{array}{cc}     4^n&0\\0&2^n    \end{array}\right]$\\
$A^n=PJ^nP^{-1}=\left[ \begin{array}{cc}    1 &1\\0& 1   \end{array}\right]\left[ \begin{array}{cc}     4^n&0\\0&2^n    \end{array}\right]\left[ \begin{array}{cc}    1 &-1\\0& 1   \end{array}\right]=\left[ \begin{array}{cc}    4^n &2^n\\0& 2^n   \end{array}\right]\left[ \begin{array}{cc}    1 &-1\\0& 1   \end{array}\right]=\left[ \begin{array}{cc}    4^n &2^n-4^n\\0&2^n   \end{array}\right]$\\

\pagebreak
%###################                 3.4.2               #################################%
\subsection*{Zadanie 3.4.2} {\color{darkgray}
	Obliczyć $A^n$ dla macierzy\\
	$A=\left[ \begin{array}{cc} 2&1\\-4&0\end{array}\right]$\\
}\lineh
\\\\
$A^n=PJ^nP^{-1}$\\
$| [A-\lambda I] |=(-\lambda)(2-\lambda)+4=\lambda^2-2\lambda+4=0$\\
$\Delta=4-16=-12$\\
$\lambda=1\pm i\sqrt{3}$\\
\\
$\boxed{\lambda=1+ i\sqrt{3}}$\\
\\
$[A-\lambda I]\omega_1=0$\\
\\
$\left[ \begin{array}{cc}     2-1-i\sqrt{3}&1\\-4&-1-i\sqrt{3}    \end{array}\right]\left[ \begin{array}{c}     \omega_{1}\\\omega_{2}    \end{array}\right]=\left[ \begin{array}{c}     0\\0    \end{array}\right]$\\
$\begin{array}{ll}    (1-i\sqrt{3})\omega_1+\omega_2=0 &\Rightarrow \omega_2=-(1-i\sqrt{3})\omega_1 \\ -4\omega_1+(-1-i\sqrt{3})\omega_2=0&\Rightarrow -4\omega_1+4\omega_1=0\end{array}$\\
$\ \ \ \ \ \omega=\left[ \begin{array}{c}     1\\-1+i\sqrt{3}    \end{array}\right]=\left[ \begin{array}{c}   1\\-1   \end{array}\right]+i\left[ \begin{array}{c}   0\\\sqrt{3}   \end{array}\right]$\\
\\\\
$P=\left[ \begin{array}{cc}    1&0\\-1&\sqrt{3}   \end{array}\right]$\\
$P^{-1}=\frac{1}{\sqrt{3}}\left[ \begin{array}{cc}    \sqrt{3}&0\\1&1   \end{array}\right]=\left[ \begin{array}{cc}   1&0\\ \frac{\sqrt{3}}{3}&\frac{\sqrt{3}}{3} \end{array}\right]$\\
\boxed{\begin{aligned}
	\lambda=a \pm ib\\
	tg \varphi=\frac{b}{a} \Rightarrow \varphi = arctg \frac{b}{a}\\
	J=\left[ \begin{array}{cc}     a&b\\-b&a    \end{array}\right] \rightarrow J^n=(\sqrt{a^2+b^2})^n \left[ \begin{array}{cc}     \cos n\varphi&\sin n \varphi\\-\sin n \varphi&\cos n \varphi    \end{array}\right]
\end{aligned}}\\
$\varphi=arctg \sqrt{3}=\frac{\pi}{3}$\\
$J^n=2^n\left[ \begin{array}{cc}     \cos n\frac{\pi}{3}&\sin n \frac{\pi}{3}\\-\sin n \frac{\pi}{3}&\cos n \frac{\pi}{3}    \end{array}\right]$\\
\\
$A^n=\left[ \begin{array}{cc}    1&0\\-1&\sqrt{3}   \end{array}\right]\cdot 2^n \cdot \left[ \begin{array}{cc}     \cos n\frac{\pi}{3}&\sin n \frac{\pi}{3}\\-\sin n \frac{\pi}{3}&\cos n \frac{\pi}{3}    \end{array}\right] \cdot \left[ \begin{array}{cc}   1&0\\ \frac{\sqrt{3}}{3}&\frac{\sqrt{3}}{3} \end{array}\right]$\\
$=2^n \left[ \begin{array}{cc}
	\cos+\frac{\sqrt{3}}{3}\sin   &  \frac{\sqrt{3}}{3} \sin \\
	-\frac{4}{3} \sqrt{3} \sin       &   -\frac{\sqrt{3}}{3} \sin + \cos
 \end{array}\right]$\\

\pagebreak
%###################                 3.4.3               #################################%
\subsection*{Zadanie 3.4.3} {\color{darkgray}
	Obliczyć $A^n$ dla macierzy\\
	$A=\left[ \begin{array}{cc} 3&-1\\-1&3\end{array}\right]$\\
}\lineh
\\\\
$A^n=PJ^nP^{-1}$\\
$|A- \lambda I|=0$\\
$\left|\begin{array}{cc} 3-\lambda&-1\\-1&3-\lambda\end{array}\right|=(\lambda-3)^2-1=(\lambda-3-1)(\lambda-3+1)=(\lambda-4)(\lambda-2)=0$\\
$\begin{array}{cc}
\lambda_1=4 &\lambda_2=2\\
	(A-\lambda_1I)\omega_1=0		& (A-\lambda_1I)\omega_2=0 \\
	     \left[ \begin{array}{cc} -1&-1\\-1&-1\end{array}\right]  \left[ \begin{array}{c}\omega_{11}\\\omega_{12}\end{array}\right]
	&   \left[ \begin{array}{cc} 1&-1\\-1&1\end{array}\right]    \left[ \begin{array}{c}\omega_{21}\\\omega_{22}\end{array}\right]
	\\ -\omega_{11}-\omega_{12}=0	&\omega_{21}=\omega_{22}\\
	\omega_{12}=-\omega_{11}& \\
	\omega_1=\left[ \begin{array}{c}1\\-1\end{array}\right] 	&\omega_2=\left[ \begin{array}{c}1\\1\end{array}\right]
\end{array}$\\
${\color{lightgray}\boxed{      \frac{4^n}{2}=\frac{4^n}{4^\frac{1}{2}}=4^{n-\frac{1}{2}}=\frac{2^n}{2}=2^{n-1}      }}$\\
\\
$J=\left[ \begin{array}{cc}     4&0\\0&2    \end{array}\right]$\\
$J^n=\left[ \begin{array}{cc}     4^n&0\\0&2^n    \end{array}\right]$\\
$P=\left[ \begin{array}{cc}     1&1\\-1&1    \end{array}\right]$\\
$P^{-1}=\left[ \begin{array}{cc}     \frac{1}{2}&-\frac{1}{2}\\\frac{1}{2}&\frac{1}{2}    \end{array}\right]$\\
\\
$A^n=\left[ \begin{array}{cc}     1&1\\-1&1    \end{array}\right]
 \left[ \begin{array}{cc}     4^n&0\\0&2^n    \end{array}\right]
\left[ \begin{array}{cc}     \frac{1}{2}&-\frac{1}{2}\\\frac{1}{2}&\frac{1}{2}    \end{array}\right]
=
\left[ \begin{array}{cc}     4^n&2^n\\-4^n&2^n    \end{array}\right]
\left[ \begin{array}{cc}     \frac{1}{2}&-\frac{1}{2}\\\frac{1}{2}&\frac{1}{2}    \end{array}\right]
=
\left[ \begin{array}{cc}     
	4^{n-\frac{1}{2}}+2^{n-1}  &
	-4^{n-\frac{1}{2}}+2^{n-1} \\
	-4^{n-\frac{1}{2}}+2^{n-1} &
	4^{n-\frac{1}{2}}+2^{n-1}
\end{array}\right]
$\\















\pagebreak
%###################                 3.5.1               #################################%
\subsection*{Zadanie 3.5.1} {\color{darkgray}
	Do ciągłego systemu dynamicznego opisanego równaniami\\
	$\dot{x}_1(t)=2\pi x_2(t)$,\\
	$\dot{x}_2(t)=-2 \pi x_1(t)+u(t)$,\\
	podłączono ekstrapolator rzędu zerowego na wejściu i impulsator na wyjściu, przy czym pracują one synchronicznie z okresem próbkowania $h = 1 s$. Wyliczyć parametry systemu dyskretnego odpowiadające takiemu połączeniu.\\
}\lineh
\\\\
$\dot{x}_1(t)=\left[ \begin{array}{cc}     0&2\pi\\-2\pi&0    \end{array}\right]_Ax(t)+\left[ \begin{array}{c}     0\\1    \end{array}\right]_Bu(t)$\\
$x^+(vi)=A^+x^+(i)+B^+u^+(i), h=1s$\\
$A^+=e^{hA}$\\
$A^+=e^{A}=Pe^JP^{-1}$\\
$\left[ \begin{array}{cc}     -\lambda&2\pi\\-2pi&-\lambda    \end{array}\right]=0 \Leftrightarrow \begin{array}{l}\lambda^2+4\pi^2=0 \\ \lambda=-4\pi^2 \\ \lambda_{1,2}=\pm 2\pi i\end{array} \begin{array}{l} \\ \\{\color{darkgray} (p=0, q=2\pi)} \end{array}$\\
$\left[ \begin{array}{cc}     -2\pi i& \boxed{2\pi } \\-2\pi&-2\pi i    \end{array}\right]
\left[ \begin{array}{c}   \omega_{11}  \\ \omega_{12}   \end{array}\right] =0$\\
$\begin{array}{r}\omega_{11} \in \mathbb{R}\\
-2\pi i \omega_{11}+2\pi \omega_{12}=0\\
\omega_{12}=i\omega_{11}\end{array} \Rightarrow \omega_1=\left[ \begin{array}{c}     \omega_{11}\\\omega_{12}    \end{array}\right]=
\omega_{11}\left[ \begin{array}{c}     1\\i    \end{array}\right]=
\omega_{11}\left[ \begin{array}{c}     1\\0    \end{array}\right]+ i\omega_{11}\left[ \begin{array}{c}     0\\1    \end{array}\right]$\\
$\Rightarrow P=\left[ \begin{array}{cc}     1&0\\0&1    \end{array}\right]=P^{-1}=I$\\
$\Rightarrow A$ jest postaci Jordana $\Rightarrow J=A=\left[ \begin{array}{cc}     0&2\pi\\-2\pi&0    \end{array}\right]$\\
$e^J=e^P \cdot\left[ \begin{array}{cc}     \cos(q)&\sin(q)\\-\sin(q)& \cos(q)   \end{array}\right]$\\
$e^J= \cdot\left[ \begin{array}{cc}     \cos(2\pi)&\sin(2\pi)\\-\sin(2\pi)& \cos(2\pi)   \end{array}\right]=\left[ \begin{array}{cc}     1&0\\0& 1   \end{array}\right]$\\
$Pe^JP^{-1}=\left[ \begin{array}{cc}     1&0\\0& 1   \end{array}\right]$\\
$B^+=\int^n_0e^{tA}B \ dt$\\
$B$ jest stałą\\
$B^+=\int^n_0e^{tA} \ dt \ B$\\
$\left| \begin{array}{c}tA=u-1\\t=uA \\dt=duA^{-1} \end{array}\right| B^+=\int_o^{hA}e^u\ du\ A^{-1}B=[e^{hA}-e^0]A^{-1}B$\\
$e^{hA}=e^A=\left[ \begin{array}{cc}     1&0\\0& 1   \end{array}\right]$\\
$e^0=1=J=\left[ \begin{array}{cc}     1&0\\0& 1   \end{array}\right]$\\
$\Rightarrow B^+=0=\left[ \begin{array}{c}    0\\0    \end{array}\right]$\\
$x^+(i+1)=A^+x^+(i)+B^+u^+(i)$\\
$x^+(i+1)=\left[ \begin{array}{cc}    1&0\\0&1    \end{array}\right]x^+(i)$\\


\pagebreak
%###################                 3.5.2               #################################%
\subsection*{Zadanie 3.5.2} {\color{darkgray}
	Do ciągłego systemu dynamicznego opisanego równaniami\\
	$\dot{x}(t)=Ax(t)+Bu(t)$,\\
	$y(t)=Cx(t)$,\\
	przy czym\\
	$A=\left[ \begin{array}{ccc}     -0.5&0&0\\0&-1&0\\0&0&-2    \end{array}\right], \ \ 
	B=\left[ \begin{array}{c}     1\\1\\1    \end{array}\right], \ \ 
	C=\left[ \begin{array}{ccc}     1&4&5\\3&1&9    \end{array}\right]$\\
	podłączono ekstrapolator rzędu zerowego na wejściu i impulsator na wyjściu, przy czym pracują one synchronicznie z okresem próbkowania $h = 1 s$. Wyliczyć parametry systemu dyskretnego odpowiadające takiemu połączeniu.\\
}\lineh
\\\\
$x^+(i)=x(ih)=x(i)$\\
$y^+(i)=y(ih)=y(i)$\\
$u^+(i)=u(ih)=u(i)$\\
$A^+=e^{hA}$\\
A jest w postaci Jordana, więc $A=J$\\
$A^+=e^{hJ}=e^J=\left[ \begin{array}{ccc}     e^{-0.5}&0&0\\0&e^{-1}&0\\0&0&e^{-2}    \end{array}\right]$\\
$B^+=\int_0^he^{tA}B \ dt = \int_0^1\left[ \begin{array}{ccc}     e^{-0.5}&0&0\\0&e^{-1}&0\\0&0&e^{-2}    \end{array}\right]   
\left[ \begin{array}{c}     1\\1\\1    \end{array}\right] dt
= \int_0^1\left[ \begin{array}{c}   e^{-\frac{t}{2}} \\e^{-t} \\ e^{-2t}   \end{array}\right] dt$\\
$=\left[ \begin{array}{r}   -2e^{-\frac{t}{2}}|_0^1 \\-e^{-t}|_0^1 \\-\frac{1}{2} e^{-2t}|_0^1   \end{array}\right]
=\left[ \begin{array}{c}    2-2e^{-\frac{1}{2}}  \\  1-e^{-1} \\\frac{1}{2}-\frac{1}{2}e^{-2}  \end{array}\right]$\\
$C^+=C=\left[ \begin{array}{ccc}     1&4&5\\3&1&9    \end{array}\right]$\\


\pagebreak
%###################                 3.5.3               #################################%
\subsection*{Zadanie 3.5.3} {\color{darkgray}
	Do ciągłego systemu dynamicznego opisanego równaniami\\
	$\dot{x}_1(t)=-2x_1(t)+x_2(t)$,\\
	$\dot{x}_2(t)=-2x_1(t)$,\\
	podłączono ekstrapolator rzędu zerowego na wejściu i impulsator na wyjściu, przy czym pracują one synchronicznie z okresem próbkowania $h = 1 s$. Wyliczyć parametry systemu dyskretnego odpowiadające takiemu połączeniu.\\
}\lineh
\\\\
$A=-2\ \ \ \ B=1\ \ \ \ \ C=-2$\\
$A^+=e^{Ah}=e^{-2}$\\
$B^+=\int_0^he^{tA}B\ dt=\int_0^he^{-2t}\ dt=|-\frac{1}{2}e^{-2t}|^h_0=-\frac{1}{2}e^{-2}$\\
$C^+=C=-2$\\
$\boxed{A^+=e^-2\ \ \ \ \ \ \ B^+=-\frac{1}{2}e^{-2}\ \ \ \ \ \ \ C^+=-2}$
































\pagebreak
%###################                 3.6.1               #################################%
\subsection*{Zadanie 3.6.1} {\color{darkgray}
	Dla jakich wartości parametrów $k_1$ i $k_2$ system dynamiczny\\
	$x(k+1)=\left[ \begin{array}{cc}     0&1\\-k_1&k_2    \end{array}\right]x(k)$\\
	będzie asymptotycznie stabilny. Zaznaczyć obszar stabilności na płaszczyźnie $k1 \times k2$\\
}\lineh
\\\\
$x(k+1)=Ax(k)$\\
$| [A-\lambda I] | =0$\\
$(-\lambda)(k_2-\lambda)+k_1=0$\\
$\lambda^2-k_2\lambda+k_1=0$\\
$\lambda=\frac{z+1}{z-1}$\\
$(\frac{z+1}{z-1})^2-k_2(\frac{z+1}{z-1})+k_1=0$\\
$\frac{(z+1)^2-k_2(z+1)(z-1)+k_1(z-1)^2}{(z-1)^2}=0$\\
Rozważany układ jest asymptotycznie stabilny $\Leftrightarrow$ wartości własne ($\lambda$) macierzy $A$ leżą wewnątrz koła jednostkowego na płaszczyźnie zespolonej $\Leftrightarrow$ pierwiastki wielomianu $L(z)=(z+1)^2-k_2(z+1)(z-1)+k_1(z-1)^2$
leżą w lewej półpłaszczyźnie zespolonej $\Leftrightarrow$ spełnione jest dla wielomianu $L(z)$ kryterium Hurwitza\\
$L(z)=(z+1)^2-k_2(z+1)(z-1)+k_1(z-1)^2$\\
$L(z)=x^2+2z+1-k_2z^2+k_2+k_1z^2-2k_1z+k_1$\\
$L(z)=z^2(1+k_1-k_2)+z(2-2k_1)+(1+k_1+k_2)=0$\\
\\
W.K. $a_2>0, a_1>0, a_0>0$\\
\\
$\begin{array}{rl}    
	1+k_1-k_2&>0\\
	k_2&<k_1+1\\
	2-2k_1&>0\\
	k_1&<1\\
	1+k_1+k_2&>0\\
	k_2&>-k_1-1
\end{array}$\\
\\
$\left[ \begin{array}{cc}     a_1&0\\a_2&a_0    \end{array}\right]=
\left[ \begin{array}{cc}     2-2k_1&0\\1+k_1-k_2&1+k_1+k_2    \end{array}\right]$\\
$a_1>0 \wedge a_1a_0>0$ - spełnione dla W.K.\\
\begin{figure}[!h]
\begin{tikzpicture}[scale=0.5]
	\draw[thick][->](-5,0)--(5,0) node [right=3pt]{$k_1$};
	\draw[thick][->](0,-5)--(0,5) node [right=3pt]{$k_2$};

	\foreach \y [count=\i] in {-1,...,-4} {{
	     \draw (-0.1,\y) -- (0.1,\y) node [left=2pt]{{\tiny \y}};
 	}}
	\foreach \y [count=\i] in {1,...,4} {{
	     \draw (-0.1,\y) -- (0.1,\y) node [left=2pt]{{\tiny \y}};
 	}}
	\foreach \x [count=\i] in {-1,...,-4} {{
	     \draw (\x,-0.1) -- (\x,0.1) node [below=4pt]{{\tiny \x}};
 	}}
	\foreach \x [count=\i] in {1,...,4} {{
	     \draw (\x,-0.1) -- (\x,0.1) node [below=4pt]{{\tiny \x}};
 	}}
		
	% koniec Osi
	
	\draw[dashed, draw=lightgray] (-5,-4) -- (3,4);
	\draw[dashed, draw=lightgray] (-5,4) -- (3,-4);
	\draw[dashed, fill=blue,fill opacity=0.5] (-1,0)--(1,2)--(1,-2)--cycle;

	
\end{tikzpicture}
\end{figure}

\pagebreak
%###################                 3.6.2              #################################%
\subsection*{Zadanie 3.6.2} {\color{darkgray}
	Dla jakich wartości parametrów $k_1$ i $k_2$ system dynamiczny\\
	$x(k+1)=\left[ \begin{array}{ccc}    k_1-k_2 & 1 & 2 \\ 0 & k_1+k_2 & 1 \\ 0 & 0 & k_1^2+k_2^2    \end{array}\right]x(k)$\\
	będzie asymptotycznie stabilny. Zaznaczyć obszar stabilności na płaszczyźnie $k1 \times k2$\\
}\lineh
\\\\
$\lambda_1 = k_1-k_2 \ \ \vee \  \ \lambda_2=k_1+k_2 \ \ \vee \ \ \lambda_3=k_1^2+k_2^2$\\
dyskretny system liniowy jesy asymprotycznie stabilny $\Leftrightarrow$ wartości własne macierzy $A$ leżą w kole jednostkowym o środku w zerze na płasczyźnie zespolonej (wystarczy sprawdzić warunek $|\lambda_i|<1$, nie trzeba z Hurwitza)\\
\\
$\bullet\ \ \lambda_1 = k_1-k_2 $\\
$\begin{array}{ccc}   k_1-k^2<1 & \wedge &k_1-k_2>-1 \\ k_2>k_1-1 & \wedge & k_2<k_1+1    \end{array}$\\
\\
$\bullet\ \ \lambda_1 =k_1+k_2$\\
$\begin{array}{ccc}  k_1+k_2<1 & \wedge & k_1+k_2>-1 \\ k_2<-k_1+1 & \wedge & k_2>-k_1-1    \end{array}$\\
\\
$\bullet\ \ \lambda_1 =k_1^2+k_2^2$\\
$\begin{array}{ccc}  k_1^2+k_2^2<1 &\wedge & k_1^2+k_2^2>-1  \end{array}$\\
\\
$\lambda_1 = k_1-k_2 \ \ \wedge \  \ \lambda_2=k_1+k_2 \ \ \wedge \ \ \lambda_3=k_1^2+k_2^2$\\

\begin{figure}[!h]
\begin{tikzpicture}[scale=0.7]
	\draw[thick][->](-5,0)--(5,0) node [right=3pt]{$k_1$};
	\draw[thick][->](0,-5)--(0,5) node [right=3pt]{$k_2$};

	\foreach \y [count=\i] in {-1,...,-4} {{
	     \draw (-0.1,\y) -- (0.1,\y) node [left=2pt]{{\tiny \y}};
 	}}
	\foreach \y [count=\i] in {1,...,4} {{
	     \draw (-0.1,\y) -- (0.1,\y) node [left=2pt]{{\tiny \y}};
 	}}
	\foreach \x [count=\i] in {-1,...,-4} {{
	     \draw (\x,-0.1) -- (\x,0.1) node [below=4pt]{{\tiny \x}};
 	}}
	\foreach \x [count=\i] in {1,...,4} {{
	     \draw (\x,-0.1) -- (\x,0.1) node [below=4pt]{{\tiny \x}};
 	}}
		
	% koniec Osi
	
	\draw[dashed, draw=lightgray] (-5,-4) -- (3,4);
	\draw[dashed, draw=lightgray] (-5,4) -- (3,-4);
	\draw[dashed, draw=lightgray] (5,-4) -- (-3,4);
	\draw[dashed, draw=lightgray] (5,4) -- (-3,-4);
	\draw[dashed, fill=blue,fill opacity=0.5] (-1,0)--(0,1)--(1,0)--(0,-1)--cycle;

	
\end{tikzpicture}
\end{figure}

\pagebreak
%###################                 3.6.3               #################################%
\subsection*{Zadanie 3.6.3} {\color{darkgray}
	Dla jakich wartości parametrów $k_1$ i $k_2$ system dynamiczny\\
	$x(k+1)=\left[ \begin{array}{cc}    -k_2&k_1\\-k_1&-k_2    \end{array}\right]x(k)$\\
	będzie asymptotycznie stabilny. Zaznaczyć obszar stabilności na płaszczyźnie $k1 \times k2$\\
}\lineh
\\\\
$|A- \lambda I|=0$\\
$\left[ \begin{array}{cc}    -k_2-\lambda&k_1\\-k_1&-k_2-\lambda    \end{array}\right]=(k_2+\lambda)^2+k_1^2=(k_2+\lambda)^2-(ik_1)^2=(\lambda +k_2+ik_1)\cdot (\lambda+k_2-ik_1)=0$\\
\\
$\lambda=-k_2 \pm k_1$\\
Układ jest asymptotycznie stabilny, gdy wartości własne macierzy leżą wewnątrz koła jednostkowego na płaszczyźnie zespolonej o środku w zerze.\\
$|\lambda |<1$\\
$|\lambda|=\sqrt{(-k_2)^2+k_1^2}$\\
$\sqrt{k_1^2+k_2^2}<1$\\
$k_1^2+k_2^2<1$\\

\begin{figure}[!h]
\begin{tikzpicture}
	\draw[thick][->](-3,0)--(3,0) node [right=3pt]{$k_1$};
	\draw[thick][->](0,-3)--(0,3) node [right=3pt]{$k_2$};

	\foreach \y [count=\i] in {-1,...,-2} {{
	     \draw (-0.1,\y) -- (0.1,\y) node [left=2pt]{{\tiny \y}};
 	}}
	\foreach \y [count=\i] in {1,...,2} {{
	     \draw (-0.1,\y) -- (0.1,\y) node [left=2pt]{{\tiny \y}};
 	}}
	\foreach \x [count=\i] in {-1,...,-2} {{
	     \draw (\x,-0.1) -- (\x,0.1) node [below=4pt]{{\tiny \x}};
 	}}
	\foreach \x [count=\i] in {1,...,2} {{
	     \draw (\x,-0.1) -- (\x,0.1) node [below=4pt]{{\tiny \x}};
 	}}
		
	% koniec Osi
	
	\draw[dashed, fill=blue,fill opacity=0.5] (0,0) circle (1);
\end{tikzpicture}
\end{figure}












\pagebreak
%###################                 3.7.1              #################################%
\subsection*{Zadanie 3.7.1} {\color{darkgray}
	Dla jakich wartości parametru $k_1$ system dynamiczny\\
	$x(k+1)=\left[ \begin{array}{cc}     -0.5k_1&0\\2&-k_1    \end{array}\right]x(k)$\\
	będzie niestabilny.\\
}\lineh
\\\\
$(-0.5k_1-2)(-k_1-\lambda)=0$\\
$0.5k_1^2+0.5k_1\lambda+k_1\lambda+\lambda^2=0$\\
$\lambda^2+1.5k_1\lambda+0.5k_1^2=0$\\
$\lambda=\frac{z+1}{z-1}$\\
$(\frac{z+1}{z-1})^2-1.5k_1(\frac{z+1}{z-1})+0.5k_1^2=0$\\
$z^2+2z+1+1.5k_1z^2-1.5k_1+0.5k_1^2z^2-k_1^2z+0.5k_1^2=0$\\
$z^2\underbrace{(0.5k_1^2+1.5k_1+1)}_{a_2}+z\underbrace{(2-k_1^2)}_{a_1}+\underbrace{(0.5k_1^2-1.5k_1+1)}_{a_0}=0$\\\\
Z kryterium Hurwitza system będzie stabilny gdy:\\
 $\begin{array}{c}   a_2>0 \\ a_1>0 \\ a_0>0    \end{array} 
\Leftrightarrow  \begin{array}{r} 
	0.5k_1^2+1.5k_1+1>0 \\ 
	k_1^2+3k_1+2>0 \\
 	(k_1+1)(k_1+2)>0 
\end{array}   
\vee \begin{array}{r}
	2-k_1^2>0 \\
	(\sqrt{2}-k_1)(\sqrt{2}+k_1)>0\\
	(k_1-\sqrt{2})(\sqrt{2}+k_1)<0
\end{array}$
\begin{figure}[!h]
\begin{tikzpicture}[scale=0.5]
	\draw[thick][->](-4,0)--(4,0) node [right=3pt]{$k$};

	\draw(-3.0,2.1)--(-2.5,1.0)--(-2.0,0.1)--(-1.5,-0.6)--(-1.0,-1.1)--(-0.5,-1.4)
			--(0.0,-1.5)--(0.5,-1.4)--(1.0,-1.1)--(1.5,-0.6)--(2.0,0.1)--(2.5,1.0)--(3.0,2.1);
	\filldraw[draw opacity=0, fill=blue,fill opacity=0.2](-3.0,2.1)--(-2.5,1.0)--(-2.0,0.1)--(-1.5,-0.6)--(-1.0,-1.1)--(-0.5,-1.4)
			--(0.0,-1.5)--(0.5,-1.4)--(1.0,-1.1)--(1.5,-0.6)--(2.0,0.1)--(2.5,1.0)--(3.0,2.1)--(3,-2)--(-3,-2)--cycle;

	\draw (-2,-0.1) -- (-2,0.1) node [below=4pt]{{\tiny -2}};
	\draw (2,-0.1) -- (2,0.1) node [below=4pt]{{\tiny -1}};		
\end{tikzpicture}
\begin{tikzpicture}[scale=0.5]
	\draw[thick][->](-4,0)--(4,0) node [right=3pt]{$k$};

	\draw(-3.0,2.1)--(-2.5,1.0)--(-2.0,0.1)--(-1.5,-0.6)--(-1.0,-1.1)--(-0.5,-1.4)
			--(0.0,-1.5)--(0.5,-1.4)--(1.0,-1.1)--(1.5,-0.6)--(2.0,0.1)--(2.5,1.0)--(3.0,2.1);
	\filldraw[draw opacity=0, fill=blue,fill opacity=0.2](-3.0,2.1)--(-2.5,1.0)--(-2.0,0.1)--(-1.5,-0.6)--(-1.0,-1.1)--(-0.5,-1.4)
			--(0.0,-1.5)--(0.5,-1.4)--(1.0,-1.1)--(1.5,-0.6)--(2.0,0.1)--(2.5,1.0)--(3.0,2.1)--(3,2)--(-3,2)--cycle;

	\draw (-2,-0.1) -- (-2,0.1) node [below=4pt]{{\tiny $-\sqrt{2}$}};
	\draw (2,-0.1) -- (2,0.1) node [below=4pt]{{\tiny $\sqrt{2}$}};		
\end{tikzpicture}
\end{figure}
ponieważ W.K. zawiera w sobie W.W., aby spełnione zostało kryterium Hurwitza musi być spełniony iloczyn warunków: $a_2>0 \wedge a_1>0$

\begin{figure}[!h]
\begin{tikzpicture}
	\draw[thick][->](-4,0)--(4,0) node [right=3pt]{$k_1$};

	\draw(-4,1)--(-2.3,1)--(-2,0.1);
	\draw(-1.4,0.1)--(-1.1,1)--(1.1,1)--(1.4,0.1);
	\draw(-1,0.1)--(-0.55,1.5)--(4,1.5);

	\draw (-2,0) circle(3pt) node [below=4pt]{{\tiny $-2$}};
	\draw (-1.4,0) circle(3pt) node [below=4pt]{{\tiny -$\sqrt{2}$}};	
	\draw (-1,0) circle(3pt) node [below=4pt]{{\tiny $-1$}};	
	\draw (1.4,0) circle(3pt) node [below=4pt]{{\tiny $\sqrt{2}$}};
\end{tikzpicture}
\end{figure}
\noindent
$\Rightarrow$ układ jest stabilny $\Leftrightarrow k_1 \in (-1,\sqrt{2})$ \\
$\Rightarrow$ układ jest niestabilny $\Leftrightarrow k_1 \in (-\infty,-1) \cup (\sqrt{2},+\infty)$ \\

\pagebreak
%###################                 3.7.2              #################################%
\subsection*{Zadanie 3.7.2} {\color{darkgray}
	Dla jakich wartości parametru $k_1$ system dynamiczny\\
	$x(k+1)=\left[ \begin{array}{cc}    -1 &-k_1 \\ -k_1 &-3    \end{array}\right]x(k)$\\
	będzie niestabilny.\\
}\lineh
\\\\
$\left| \begin{array}{cc}    -1 -\lambda&-k_1 \\ -k_1 &-3-\lambda    \end{array}\right|=(1+\lambda)(3+\lambda)-k_1^2=\lambda^2+3+4\lambda-k_1^2=0$\\
$\Delta=16-12+4k_1^2=4+4k_1^2$\\
$\lambda=-2 \pm \sqrt{1+k_1^2}$\\
\\
system będzie niestabilny $\Leftrightarrow |\lambda_i|>1$\\
\\
$\bullet\ \ \lambda=-2 + \sqrt{1+k_1^2}$\\
$ \begin{array}{ccc}
-2 + \sqrt{1+k_1^2}>1                &\vee&    -2 + \sqrt{1+k_1^2}<-1  \\
\sqrt{1+k_1^2}>3                           & &         \sqrt{1+k_1^2}<1 \\
k_1^2>8                                           & &                  k_1^2<0\\
k_1>2\sqrt{2} \vee k_1<-2\sqrt{2} & &                     \text{sprzeczne}\\
\end{array} $\\
\\\\
$\bullet\ \ \lambda=-2 - \sqrt{1+k_1^2}$\\
$ \begin{array}{ccc}
-2 - \sqrt{1+k_1^2}>1                &\vee&    -2 - \sqrt{1+k_1^2}<-1  \\
\sqrt{1+k_1^2}>-3                           & &         \sqrt{1+k_1^2}>-1 \\
\text{sprzeczne}                                & &                  k_1 \in \mathbb{R}
\end{array} $\\
\\\\
$\lambda=-2 + \sqrt{1+k_1^2} \vee \lambda=-2 - \sqrt{1+k_1^2} \Rightarrow k_1 \in \mathbb{R}$

\pagebreak
%###################                 3.7.3              #################################%
\subsection*{Zadanie 3.7.3} {\color{darkgray}
	Dla jakich wartości parametru $k_1$ system dynamiczny\\
	$x(k+1)=\left[ \begin{array}{cc}    -k_1 &k_1 \\ -k_1 &-k_1    \end{array}\right]x(k)$\\
	będzie niestabilny.\\
}\lineh
\\\\
$\left[ \begin{array}{cc}    -k_1 -\lambda&k_1 \\ -k_1 &-k_1-\lambda    \end{array}\right]
=(-k_1-\lambda)^2+k_1^2=(-k_1-\lambda)^2-(ik_1)^2=(-k_1-\lambda-ik_1)(-k_1-\lambda+ik_1)=(\lambda+k_1-ik_1)(\lambda+k_1+ik_1)=0$\\
$\lambda=-k_1 \pm ik_1$\\
układ jest niestabilny gdy $|\lambda|>1$\\
podstawiamy tu moduł - więc usuwamy i z zespolonej.\\
$|\lambda|=\sqrt{(-k_1)^2+(k_1)^2}=\sqrt{k_1^2+k_1^2}>1$\\
$2k_1^2>1$\\
$k_1^2>\frac{1}{2}$\\
$k_1>\frac{\sqrt{2}}{2}$\\













\pagebreak
%###################                 3.8.1              #################################%
\subsection*{Zadanie 3.8.1} {\color{darkgray}
	Wyznaczyć rozwiązanie następującego równania różnicowego (rekurencyjnego)\\
	$x(k+2)+x(k+1)-2x(k)=0, x(0)=1,x(1)=-1$\\
}\lineh
\\\\
zakładam rozwiązanie postaci $z^n$\\
$x^n, +z^{n-1}+2z^{n-2}=0$\\
$z^2+z-2=0$ - wielomian charakterystyczny\\
$W(z)=(z+2)(z-1)$\\
$z_1=-2, \ \ z_2=1$\\
rozwiązanie równania jest postaci:\\
$x(k)=Cz_1^k+Dz_2^k$\\
podstawiając $x(0)$ oraz $x(1)$\\
$\begin{cases} 
	x(0)=1=C+D \\
	x(1)=-1=Cz_1+Dz_2
\end{cases}$\\
$\begin{cases}
	 C=1-D \\
	-1=-2C+D
\end{cases}$\\
$\begin{cases}
	 D=1-C\\
	-1=-2C+1-C
\end{cases}$\\
$\begin{cases}
	 D=1-C\\
	3C=2
\end{cases}$\\
$C=\frac{2}{3} \ \ \ D=\frac{1}{3}$\\
$\Rightarrow x(k)=\frac{1}{3}(2 \cdot(-2)^k+1^k)$\\
$\boxed{x(k)=\frac{2}{3}(-2)^k+\frac{1}{3}1^k}$

\pagebreak
%###################                 3.8.2              #################################%
\subsection*{Zadanie 3.8.2} {\color{darkgray}
	Wyznaczyć rozwiązanie następującego równania różnicowego (rekurencyjnego)\\
	$x(k+2)+3x(k+1)+2x(k)=0, x(0)=2,x(1)=-3$\\
}\lineh
\\\\
$q^{k+2}+3q^{k+1}+2q=0$\\
$q^2+3q+2=0$\\
$\Delta=9-8=1$\\
$q=\frac{-3+1}{2}=-1 \ \ \vee \ \ q=\frac{-3-1}{2}=-2$\\
$x(k)=a(-1)^k+b(-2)^k$\\
$\begin{cases} x(0)=a+b=2 \\ x(1)=-a-2b=-3\end{cases}$\\
$b=1 \wedge a=1$\\
$\boxed{x(k)=(-1)^k+(-2)^k}$

\pagebreak
%###################                 3.8.3              #################################%
\subsection*{Zadanie 3.8.3} {\color{darkgray}
	Wyznaczyć rozwiązanie następującego równania różnicowego (rekurencyjnego)\\
	$x(k+2)+2x(k+1)-3x(k)=0, x(0)=2,x(1)=-2$\\
}\lineh
\\\\
$x^2+2x-3=0$\\
$\Delta=4+12=16, \sqrt{\Delta}=4$\\
$x_1=\frac{-2-4}{2}=-3 \wedge x_2=1$\\
$x(k)=C_1x_1^k+C_2x_2^k$\\
$\begin{cases} x(0)=2=C_1+C_2 \Rightarrow C_2=2-C_1 \\ x(1)=-2=-3C_1+C_2\end{cases}$\\
$-2=-3C_1+2-C_1 \Rightarrow C_1=1$\\
$C_2=1$\\
$\boxed{x(k)=-3^k+1}$\\

\noindent{\color{red}Alternatywne : }
\\
równanie charakterystyczne\\
$r^2+2r-3=0$\\
$(r+1)^2-4=0$\\
$(r+1-2)(r+1+2)=0$\\
$(r-1)(r+3)$\\
$r_1=1\ \ \ \ \ r_2=-3$\\
\\
$x(k)=c_1r_1^k+c_2r_2^k$\\
$x(0)=2=c_1+c_2$\\
$x(1)=-2=c_1+3c_2$\\
\\
$\begin{cases}2=c_1+c_2\\-2=c_1-3c_2 \end{cases}$\\
$4=4c_2$\\
$c_1=1\ \ \ \ c_2=1$\\
\\
$\boxed{x(k)=1+(-3)^k}$










\pagebreak
%###################                 3.9.1              #################################%
\subsection*{Zadanie 3.9.1} {\color{darkgray}
	Dany jest układ dyskretny\\
	$x(k+1)=Ax(k), A=\left[ \begin{array}{ccccc}    0&\cdots&0&0&0 \\  1&\cdots&0&0&0 \\\vdots&\ddots&\vdots&\vdots&\vdots \\ 1&\cdots&1&0&0    \\1&\cdots&1&1&0\end{array}\right]_{n \times n}$\\
	przy czym $k=1,2,...$ Wyznaczyć $x(n)$.\\
}\lineh
\\\\
wielomian charakterystyczny: $| [A-\lambda I] |=0$\\
$\left[ \begin{array}{ccccc}    -\lambda&\cdots&0&0&0 \\  1&\cdots&0&0&0 \\\vdots&\ddots&\vdots&\vdots&\vdots \\ 1&\cdots&1&-\lambda&0    \\1&\cdots&1&1&-\lambda\end{array}\right]=(-\lambda)^n${\color{lightgray} (wyznacznik macierzy diagonalnej)}\\\\
Z Tw. Cagleya-Hamiltona wiadomo, że każda macierz spełnia swój wielomian charakterystyczny\\
$A^n=0 $\\\\
rozwiązanie równania $x(k+1)=Ax(k)$ ma postać $x(k)=A^kx(0)$ czyli $x(n)=A^nx(0)=0$\\
Biorąc więc pod uwagę fakt $A^n=0 $ wiadomo, że rozwiązanie x(k) stanie się zerem w cco najwyżej n krokach, niezależnie od warunku początkowego $x(0)$

\pagebreak
%###################                 3.9.2              #################################%
\subsection*{Zadanie 3.9.2} {\color{darkgray}
	Dany jest układ dyskretny\\
	$x(k+1)=Ax(k)+Bu(k), A=\left[ \begin{array}{ccccc}   
		 0&1&2&\cdots&n-1 \\
		 0&0&1&\cdots&n-2  \\
		 \vdots&\vdots&\vdots&\ddots&\vdots  \\
		 0&0&0&\cdots&1  \\
		 0&0&0&\cdots&0  \\
	\end{array}\right]_{n \times n}, \ \ \ \ B=\left[\begin{array}{c}   
		1 \\ 0 \\ \vdots \\ 0 \\ 0
	\end{array}\right]$\\
	przy czym $k=1,2,...$ Wyznaczyć $x(n)$ wiedząc, że $u(i)=1$ dla $i=1,2,...,n$.\\
}\lineh
\\\\
$x(n)=A^nx(0)+\sum_{j=0}^{n-1}A^{n-1-j}Bu(j)$\\
Zauważmy, że $\det (A) = 0$ (bo same zera na przekątnej)\\
wtedy $\det(\lambda I-A)=\lambda^n$\\
z Tw. Cagleya-Hamiltona $A^n=0$ (każda macierz spełnia swój wielomian charakterystyczny)\\
mamy więc $A^nx(0)=0$, czyli $x(n)=\sum_{j=0}^{n-1}A^{n-1-j}B\underbrace{u(j)}_{=1}$\\
Przy kolejnych mnożeniach macierzy $A$ podniesionej do jakiejś potęgi przez $B$ otrzymujemy pierwszą kolumnę macierzy $A$, która zawiera same zera, poza przypadkiem $A^0=I$ (macierz jednostkowa), więc sumujemy $n-1$ kolumn samych zer oraz jedną równą $B \  (I\cdot B=B)$\\
ostatecznie $\boxed{x(n)=B}$

\pagebreak
%###################                 3.9.3              #################################%
\subsection*{Zadanie 3.9.3} {\color{darkgray}
	Dany jest układ dyskretny\\
	$x(k+1)=Ax(k)+Bu(k), A=\left[ \begin{array}{ccccc}   
		 0&1&1&\cdots&1 \\
		 0&0&1&\cdots&1  \\
		 \vdots&\vdots&\vdots&\ddots&\vdots  \\
		 0&0&0&\cdots&1  \\
		 0&0&0&\cdots&0  \\
	\end{array}\right]_{n \times n}, \ \ \ \ B=\left[\begin{array}{c}   
		1 \\ 0 \\ \vdots \\ 0 \\ 0
	\end{array}\right]$\\
	przy czym $k=1,2,...$ Wyznaczyć $x(n)$ wiedząc, że $u(i)=1$ dla $i=1,2,...,n$.\\
}\lineh
\\\\
$|A|=0$\\
$x(k+1)=Ax(k)$\\
$|\lambda I -A|=\lambda^n$\\
z Tw. Cagleya-Hamiltona:\\
$A^n=0$\\
$x(1)=Ax(0)+B$\\
$x(2)=A^2x(0)+(A+1)B$\\
$x(3)=A^3x(0)+(A^2+A+1)B$\\
$...$\\
$x(n)=A^nx(0)+(A^{n-1}+A^{n-2}+...+1)B$\\
ponieważ tylko $A^n=0$, więc dla każdego $A^{n-1}, A^{n-2},...\neq0$\\
te potęgi będą miały jakieś śmieci w wartościach, nie istotne co tam będzie. Ważne, że tam gdzie w $A$ są $0$ nie pojawi się nic nowego, czyli gdzie było $0$ przed potęgowaniem, tam będzie też po. $\Rightarrow$ wynik iloczynu $A^iB=0$ (macierz zerowa), $i=1,...,n-1 \ \ \Rightarrow \boxed{x(n)=B}$ 

\pagebreak
%###################                 3.9.4              #################################%
\subsection*{Zadanie 3.9.4} {\color{darkgray}
	Dany jest układ dyskretny\\
	$x(k+1)=Ax(k), A=\left[ \begin{array}{ccccc}   
		 0&1&1&\cdots&1 \\
		 0&0&1&\cdots&1  \\
		 \vdots&\vdots&\vdots&\ddots&\vdots  \\
		 0&0&0&\cdots&1  \\
		 0&0&0&\cdots&0  \\
	\end{array}\right]_{n \times n}$\\
	przy czym $k=1,2,...$ Wyznaczyć $x(2n)$.\\
}\lineh
\\\\
$|A|=0$\\
$x(k+1)=Ax(k)$\\
$|\lambda I -A|=\lambda^n$\\
z Tw. Cagleya-Hamiltona:\\
$A^n=0$\\
$x(1)=Ax(0)$\\
$x(2)=A^2x(0)$\\
$x(3)=A^3x(0)$\\
$...$\\
$x(n)=A^nx(0)=0$\\
$x(2n)=A^nx(n)=0$\\

\pagebreak
\section*{Tydzień 4}
Analiza częstotliwościowa systemów dynamicznych


%###################                 4.1.1              #################################%
\subsection*{Zadanie 4.1.1} {\color{darkgray}
	Układ jest opisany równaniami stanu w postaci\\\\
	$\begin{cases} \dot{x}(t)=Ax(t)+Bu(t)\\y(t)=Cx(t)\end{cases}$\\\\
	z macierzami\\\\
	$A=\left[\begin{array}{cc}1&0\\2&-3\end{array}\right],
	B=\left[\begin{array}{c}1\\0\end{array}\right],
	C=\left[\begin{array}{cc}	1&0\end{array}\right]
	$\\\\
	Znaleźć transmitancję operatorową tego układu przy założeniu zerowych warunków początkowych $x(0)=0$\\
}\lineh
\\\\
$G(s)=C(sI-A)^{-1} \cdot B$\\
$(sI-A)^{-1}=\left[\begin{array}{cc}s-1&0\\-2&s-3\end{array}\right]^{-1}=\frac{1}{(s-1)(s-3)}\left[\begin{array}{cc}s-3&0\\2&s-1\end{array}\right]=\left[\begin{array}{cc}\frac{1}{s-1}&0\\\frac{2}{(s-1)(s-3)}&\frac{1}{s-3}\end{array}\right]$\\
$G(s)=\left[\begin{array}{cc}1&0\end{array}\right]\cdot\left[\begin{array}{cc}\frac{1}{s-1}&0\\\frac{2}{(s-1)(s-3)}&\frac{1}{s-3}\end{array}\right]\cdot \left[\begin{array}{c}1\\0\end{array}\right]=\left[\begin{array}{cc}\frac{1}{s-1}&0\end{array}\right] \cdot \left[\begin{array}{c}1\\0\end{array}\right]=\frac{1}{s-1}$\\



%###################                 4.1.2              #################################%
\pagebreak
\subsection*{Zadanie 4.1.2} {\color{darkgray}
	Układ jest opisany równaniami stanu w postaci\\\\
	$\begin{cases} \dot{x}(t)=Ax(t)+Bu(t)\\y(t)=Cx(t)\end{cases}$\\\\
	z macierzami\\\\
	$A=\left[\begin{array}{cc}2&0\\-5&0\end{array}\right],
	B=\left[\begin{array}{c}2\\2\end{array}\right],
	C=\left[\begin{array}{cc}0&1\end{array}\right]$\\\\
	Znaleźć transmitancję operatorową tego układu przy założeniu zerowych warunków początkowych $x(0)=0$\\
}\lineh
\\\\
$G(s)=C(sI-A)^{-1}B$\\
$G(s)=\left[\begin{array}{cc}0&1\end{array}\right]
\cdot
\left[\begin{array}{cc}{s-2}&0\\5&s\end{array}\right]^{-1}
\cdot
\left[\begin{array}{c}2\\2\end{array}\right]=\left[\begin{array}{cc}0&1\end{array}\right]
\cdot 
\left[\begin{array}{cc}{\frac{1}{s-2}}&0\\-5&\frac{1}{s}\end{array}\right]
\cdot
\left[\begin{array}{c}2\\2\end{array}\right]=\frac{2}{s}-\frac{10}{s\cdot(s-2)}=\frac{2s-14}{s\dot{(s-2)}}
$

%###################                 4.1.3              #################################%
\pagebreak
\subsection*{Zadanie 4.1.3} {\color{darkgray}
	Układ jest opisany równaniami stanu w postaci\\\\
	$\begin{cases} \dot{x}(t)=Ax(t)+Bu(t)\\y(t)=Cx(t)\end{cases}$\\\\
	z macierzami\\\\
	$A=\left[\begin{array}{cc}2&3\\1&0\end{array}\right],
	B=\left[\begin{array}{c}2\\5\end{array}\right],
	C=\left[\begin{array}{cc}0&1\end{array}\right]$\\\\
	Znaleźć transmitancję operatorową tego układu przy założeniu zerowych warunków początkowych $x(0)=0$\\
}\lineh
\\\\


%###################                 4.2.1              #################################%
\pagebreak
\subsection*{Zadanie 4.2.1} {\color{darkgray}
	Mając dana transmitancje $G(s)=\frac{5}{s+3}$ okreslić amplitudę sygnału wyjściowego, jesli na wejście podano:\\
	a) $2\sin(4t+2 \pi)$\\
	b) $-\sin(t)$\\
	c) $0.1\cos(4t+\frac\pi 6)$\\
}\lineh
\\\\
$G(s)=\frac{5}{s+3}$\\
$A(\omega)=|G(j\omega)|$\\
$u(t)=A_u\sin(\omega t+\varphi_u )$ - wejście\\
$A_y=A(\omega) \cdot A_u$\\
\\
\textbf{a)}\\
$2\sin(4t+2\pi)=u(t)$\\
$A_u=2$\\
$\omega=4$\\
$A(\omega)=|\frac{5}{4j+3}|=|\frac{5(3-4j)}{9+16}|=|\frac{3-4j}{5}|=\sqrt{\frac{9}{25}+\frac{16}{25}}=1$\\
$A_y=1 \cdot 2 = \boxed{2}$ \ \ \ \ \ \   {\color{lightgray}$A(\omega)=\frac{5}{\sqrt{16+9}}=1$}\\
\\
\textbf{b)}\\
$-\sin(t)=u(t)$\\
$A_u=-1$\\
$\omega=1$\\
$A(\omega)=|\frac{5}{j+3}|=|\frac{5(3-j)}{9+1}|=|\frac{3-j}{2}|=\sqrt{\frac 94+\frac 14}=\frac{\sqrt{10}}{2}$\\
$A_y=-1 \cdot \frac{\sqrt{10}}{2}=\boxed{-\frac{\sqrt{10}}{2}}$\ \ \ \ \ \ \   {\color{lightgray}$A(\omega)=\frac{5}{\sqrt{9+1}}=\frac{\sqrt{10}}{2}$}\\
\\
\textbf{c)}\\
$0.1\cos(4t+\frac\pi 6)=u(t)$\\
$\cos(\frac\pi 2 - \alpha)=\sin \alpha \Rightarrow \cos(4t+\frac \pi 6 ) = \cos (\frac \pi 2 -(\frac{2\pi}{6} - 4t)= \sin(\frac{2\pi}{6}-4t)$\\
$u(t)=\frac{1}{10}\sin(\frac \pi 3 - 4t)$\\
$A_u = \frac{1}{10} \ \ \ \ \ \omega=-4$\\
$A(\omega)=|\frac{5}{-4j+3}|=|\frac{5(3+4j)}{9+16}|=|\frac{3+4j}{5}|=\sqrt{\frac{9}{25}+\frac{16}{25}}=1$\\
$A_y=1 \cdot \frac{1}{10} = \boxed{\frac{1}{10}}$ \ \ \ \ \ \   {\color{lightgray}$A(\omega)=\frac{5}{\sqrt{16+9}}=1$}\\


%###################                 4.2.2              #################################%
\pagebreak
\subsection*{Zadanie 4.2.2} {\color{darkgray}
	Mając dana transmitancje $G(s)=\frac{30}{s+2}$ okreslić amplitudę sygnału wyjściowego, jesli na wejście podano:\\
	a) $\sin(t+\frac{2 \pi}{3})$\\
	b) $0.5\sin(2t)$\\
	c) $8\cos(3t+\frac{2\pi}{3})$\\
}\lineh
\\\\

%###################                 4.3.1              #################################%
\pagebreak
\subsection*{Zadanie 4.3.1} {\color{darkgray}
	Za pomocą transmitancji znaleźć odpowiedź układu 1 na skok jednostkowy, czyli funkcję postaci:\\
	$u(t)=\begin{cases}0,t<0 \\ 1, t\geqslant 0 \end{cases}$\\
	Zakladamy, że $x(0)=0$\\
	$\dot{x}(t)=-4x(t)+8u(t)$\\
	$y(t)=x(t)$\\
}\lineh
\\\\
$\begin{cases}\dot{x}(t)=-4x(t)+8u(t)\\y(t)=x(t) \end{cases} \ \ \ \ \ \ \ \ \ x(0)=0$\\
$C=1 \ \ \ A=-4 \ \ \ \ B=8$\\
$G(s)=1 \cdot (s+4)^{-1} \cdot 8 = 8 \cdot \frac{1}{s+4}$\\
$U(s)=\frac 1 s \ \ \ $\\ tw. Laplace'a dla skoku jednopstkowego
$Y(s)=G(s) \cdot U(s)$\\
$Y(s)=\frac{8}{s+4} \cdot \frac{1}{2} = \frac{2}{s} - \frac{2}{s+4}$\\
$\frac{A}{s}+\frac{B}{s+4}=\frac{8}{s(s+4)}$\\
$A(s+4)+Bs=8$\\
$s(A+B)+4A=8$\\
$A=2 \ \ \ B=-2$\\
$\boxed{y(t)=-2e^{-4t}+2}$\\ 
$\uparrow$\\
odwrotne tw. Laplace'a\\
$\mathcal{L}\{a\}=a\frac 1 s$\\
$\mathcal{L}\{ae^{bt}\}=a\frac {1}{s-b}$\\

%###################                 4.3.2              #################################%
\pagebreak
\subsection*{Zadanie 4.3.2} {\color{darkgray}
	Za pomocą transmitancji znaleźć odpowiedź układu 1 na skok jednostkowy, czyli funkcję postaci:\\
	$u(t)=\begin{cases}0,t<0 \\ 1, t\geqslant 0 \end{cases}$\\
	Zakladamy, że $x(0)=0$\\
	$\dot{x}(t)=5x(t)-3u(t)$\\
	$y(t)=x(t)$\\
}\lineh
\\\\


%###################                 4.4.1              #################################%
\pagebreak
\subsection*{Zadanie 4.4.1} {\color{darkgray}
	Na układ o transmitancji operatorowej $G(s)=\frac{20}{s+10}$ podano sygnał sinusoidalny $2\sin(4.5t+\frac \pi 6)$. Obliczyć, jak zmieni się amplituda sygnału wyjściowego.\\
}\lineh
\\\\
$A_u=2 \ \ \ \ \omega = 4.5=\frac 9 2$\\
$A(\omega)=|\frac{20}{j \frac 9 2 + 10}|=|\frac{20(10-\frac 9 2 j)}{100+\frac{81}{4}}|=|\frac{80(10-\frac 9 2)j}{481}|=\sqrt{\frac{800^2+360^2}{481^2}}\approx 1.82$\\
 {\color{lightgray}$A(\omega)=|\frac{20}{j\frac 9 2 +10}|=\frac{20}{\sqrt{100+\frac{81}{4}}}= \frac{40}{\sqrt{481}} \approx 1.82$}
$A_y=A(\omega)\cdot 2$\\
$\frac{A_y}{A_u}=A(\omega)$\\
\\
Amplituda sygnału wejśicowego będzie ok. 1.82 razy większa niż wejściowego\\

%###################                 4.4.2              #################################%
\pagebreak
\subsection*{Zadanie 4.4.2} {\color{darkgray}
	Na układ o transmitancji operatorowej $G(s)=\frac{400}{s+30}$ podano sygnał sinusoidalny $5\sin(2t+\frac \pi 3)$. Obliczyć, jak zmieni się amplituda sygnału wyjściowego.\\
}\lineh
\\\\


%###################                 4.5.1              #################################%
\pagebreak
\subsection*{Zadanie 4.5.1} {\color{darkgray}
	Narysowac charakterystyki Nyquista dla układu opisanego transmitancja operatorowa:\\
	$G(s)=\frac{s+2}{s^2+3s+2}$\\
	Podac wzór na transmitancje widmowa tego układu (w postaci rozbicia na czesc urojona i rzeczywistą).\\
}\lineh
\\\\
$G(s)=\frac{s+2}{s^2+3s+2}=\frac{s+2}{(s+1)(s+2)}=\frac{1}{s+1}$\\
\begin{figure}[!h]
\begin{tikzpicture}
	\draw[color=red, thick](1,0) ellipse (1 and 1);
	\filldraw[draw=white, fill=white](0,0)--(2,0)--(2,2)--(0,2)--cycle;=
	\draw[color=red, draw opacity=0.1](1,0) ellipse (1 and 1);
	
	\draw[color=red] (1,-1) -- (1.2,-1.2);
	\draw[color=red] (1,-1) -- (1.2,-0.8);

	\draw[color=red!10!white] (1,1) -- (0.8,1.2);
	\draw[color=red!10!white] (1,1) -- (0.8,0.8);


	\draw[thick][->](-1,0)--(3,0) node [right=3pt]{$Re$};
	\draw[thick][->](0,-2)--(0,2) node [right=3pt]{$Im$};

	\node at(3.4,-1.2){tylko ta część bo $\omega>0$};

	\draw (-0.1,-1) -- (0.1,-1) node [left=2pt]{{-0.5}};
	\draw (-0.1,1) -- (0.1,1) node [left=2pt]{{ 0.5}};
	\draw (2,-0.1) -- (2,0.1) node [below=4pt]{{ 1}};
\end{tikzpicture}
\end{figure}
\noindent\\
$G(j\omega)=\frac{1}{j\omega+1}=\frac{1-j\omega}{1+\omega^2}=\boxed{\frac{1}{1+\omega^2}+j\frac{-\omega}{1+\omega^2}}$\\



%###################                 4.5.2              #################################%
\pagebreak
\subsection*{Zadanie 4.5.2} {\color{darkgray}
	Narysowac charakterystyki Nyquista dla układu opisanego transmitancja operatorowa:\\
	$G(s)=\frac{2s}{3s^2+3s}$\\
	Podac wzór na transmitancje widmowa tego układu (w postaci rozbicia na czesc urojona i rzeczywistą).\\
}\lineh
\\\\


%###################                 4.6.1              #################################%
\pagebreak
\subsection*{Zadanie 4.6.1} {\color{darkgray}
	\begin{figure}[!h]
	\begin{tikzpicture}
	\draw[scale=0.8, transform shape]
	(0,0) to[american current source,l=$u(t)$] 
	(0,3) to[C,l=C,i=$i_c$] (3,3) to [european resistor,l=R] (3,0) -- (0,0);
	\draw[->] (3,0.5) -- (3,2) node[right,pos=.5] {$x_1=U_R$};
	\draw (1,1) arc (240:-50:.3 and .3);
	\draw[->] (1.35,1.04)  -- (1.25,.94) ;
	\end{tikzpicture}	
	\end{figure}
	\noindent Przeanalizować układ z rysunku i znaleźć równania opisujące ten układ. Za wyjście
	przyjąć napięcie na oporniku. Znaleźć transmitancję operatorową i widmową układu\\
	Zakladamy ze: $R=1000\Omega,C=1mF$\\
}\lineh
\\\\
\noindent $\begin{cases}
u(t)-u_c(t)-u_R(t)=0\\
y(t)=u_R(t)
\end{cases}$\\\\
$\begin{cases}
u(t)-u_c(t)-RC\dot{u_c}(t)=0\\
y(t)=RC\dot{u_c}(t)
\end{cases}$\\\\
$\left.\begin{cases}
u(t)=+u_c(t)+RC\dot{u_c}(t)\\
y(t)=RC\dot{u_c}(t)
\end{cases}\right|\mathscr{L}$\\
Transformata Laplace'a\\
$\begin{cases}
U(s)=U_c(s)+sRC\cdot U_c(s)+u_c(0)\\
Y(s)=sRC\cdot U_c(s)+u_c(0)
\end{cases}$\\
$G(s)=\frac{Y(s)}{U(s)}=\frac{sRC\cdot U_c(s)+u_c(0)}{U_c(s)+sRC\cdot U_c(s)+u_c(0)}=\frac{sRC\cdot U_c(s)}{U_c(s)+sRC\cdot U_c(s)}=\frac{\cancel{U_c(s)}\cdot sRC}{\cancel{U_c(s)}\cdot (sRC+1)}$\\\\
Podstawiamy R i C\\
$G(s)=\frac{s}{s+1}$\\
$G(j\omega)=\frac{j\omega}{j\omega+1}=\frac{j\omega}{j\omega+1}\cdot \frac{1-j\omega}{1-j\omega}=\frac{j\omega+\omega^2}{1+\omega^2}=\frac{\omega^2}{\omega^2+1}+j\frac{\omega}{\omega^2+1}$



%###################                 4.7.1              #################################%
\pagebreak
\subsection*{Zadanie 4.7.1} {\color{darkgray}
	Niech będzie dany układ opisany transmitancją $G(s)$:\\
	$G(s)=\frac{s}{s^2+2s+1}$\\
	Korzystajac z kryterium Nyquista sprawdzić, czy układ zamkniety postaci 7 będzie asymptotycznie stabilny.
\begin{figure}[!h]
\begin{tikzpicture}
	\draw(0,0)--(7,0);
	\draw(1,0)--(1,-2)--(6,-2)--(6,0);
	\filldraw[draw=black, fill=white](2,-1)--(5,-1)--(5,1)--(2,1)--cycle;
	\draw[draw=black, fill=white] (1,0) circle (0.3);
	\draw(0.8,-0.2)--(1.2,0.2);
	\draw(0.8,0.2)--(1.2,-0.2);
	\draw(0.3,-0.3)--(0.6,-0.3);
	\draw(0.45,-0.15)--(0.45,-0.45);
	\draw(1.4,-0.3)--(1.7,-0.3);
	\node at (0.3,.3) {$U(s)$};
	\node at (6.3,.3) {$Y(s)$};
	\node at (3.5,0) {$G(s)$};
\end{tikzpicture}
\end{figure}

}\lineh
\\\\
$G(s)=\frac{s}{s^2+2s+1}$\\
$a_2=1 \ \ a_1 =2 \ \ a_0 = 1$\\
Układ zamknięty będzie asymptotycznie stabilny jeżeli układ otwarty będzie asymptotycznie stabilny oraz wykres charakterystyki amplitudowo-fazowej transformacji $G(s)$ nie będzie obejmował punktu (-1, 0) na płaszczyźnie zespolonej.\\
sprawdź czy układ otwarty jest asymptotycznie stabilny z Hurwitza. Wystarczy sprawdzić dla $s^2+2s+1$\\
$
\begin{array}{c}
\begin{array}{cc}a_1 & a_3\end{array}\\
\left[\begin{array}{cc}2&0\\1&1\end{array}\right]\\
\begin{array}{cc}a_0&a_2\end{array}
\end{array}
$\\
$2>0$\\
$2\cdot 1-0>0$\\
więc układ otwarty jest asymptotycznie stabilny.
\begin{figure}[!h]
\begin{tikzpicture}
	\draw[color=red, thick](1,0) ellipse (1 and 1);
	\filldraw[draw=white, fill=white](0,0)--(2,0)--(2,-2)--(0,-2)--cycle;
	\draw[color=red, draw opacity=0.1](1,0) ellipse (1 and 1);
	
	\draw[color=red!10!white] (1,-1) -- (1.2,-1.2);
	\draw[color=red!10!white] (1,-1) -- (1.2,-0.8);

	\draw[color=red] (1,1) -- (0.8,1.2);
	\draw[color=red] (1,1) -- (0.8,0.8);


	\draw[thick][->](-1,0)--(3,0) node [right=3pt]{$Re$};
	\draw[thick][->](0,-2)--(0,2) node [right=3pt]{$Im$};

	\node at(3.2,1){tylko górna część};

	\draw (-0.1,-1) -- (0.1,-1) node [left=2pt]{{-0.25}};
	\draw (-0.1,1) -- (0.1,1) node [left=2pt]{{0.25}};
	\draw (2,-0.1) -- (2,0.1) node [below=4pt]{{0.5}};
\end{tikzpicture}
\end{figure}
\noindent\\
Nie obejmuje punktu (-1,0) więc jest asymptotycznie stabilny.\\
$G(j\omega)=\frac{j\omega}{(j\omega)^2+2j\omega+1}=\frac{j\omega}{-\omega^2+2j\omega+1}=\frac{j\omega(1-\omega^2-2j\omega)}{(1-\omega^2+2j\omega)(1-\omega^2-2j\omega)}=\frac{j\omega-j\omega^3+2\omega^2}{1-2\omega^2+\omega^4+4\omega^2}=\frac{2\omega^2}{(\omega^2+1)^2}+j\frac{\omega-\omega^3}{(\omega^2+1)^2}$\\

%###################                 4.7.2              #################################%
\pagebreak
\subsection*{Zadanie 4.7.2} {\color{darkgray}
	Niech będzie dany układ opisany transmitancją $G(s)$:\\
	$G(s)=\frac{s^2+s+1}{s^3-s^2+2s-2}$\\
	Korzystajac z kryterium Nyquista sprawdzić, czy układ zamkniety postaci 7 będzie asymptotycznie stabilny.
\begin{figure}[!h]
\begin{tikzpicture}
	\draw(0,0)--(7,0);
	\draw(1,0)--(1,-2)--(6,-2)--(6,0);
	\filldraw[draw=black, fill=white](2,-1)--(5,-1)--(5,1)--(2,1)--cycle;
	\draw[draw=black, fill=white] (1,0) circle (0.3);
	\draw(0.8,-0.2)--(1.2,0.2);
	\draw(0.8,0.2)--(1.2,-0.2);
	\draw(0.3,-0.3)--(0.6,-0.3);
	\draw(0.45,-0.15)--(0.45,-0.45);
	\draw(1.4,-0.3)--(1.7,-0.3);
	\node at (0.3,.3) {$U(s)$};
	\node at (6.3,.3) {$Y(s)$};
	\node at (3.5,0) {$G(s)$};
\end{tikzpicture}
\end{figure}

}\lineh
\\\\


%###################                 4.8.1              #################################%
\pagebreak
\subsection*{Zadanie 4.8.1} {\color{darkgray}
	Rozwiązanie równania różniczkowego\\
	$\dot{x}(t)=-4x(t)+3\sin(2t)$\\
	gdzie $x(0)=0, t \geqslant 0$ ma postać\\
	$x(t)=ae^{-4t}+A\sin(2t+\varphi)$\\
	Obliczyć $A$ i $\varphi$.\\
}\lineh
\\\\
$u(t)=3\sin(2t)$\\
$A_u=3 \ \ \ \ \omega = 2 \ \ \ \ y_u=0$\\
$A=-4 \ \ \ B=3 \ \ \ \ C=1$\\
$G(s)=1\cdot(s+4)^{-1} \cdot 3=3 \cdot \frac{1}{s+4}$\\
$A(\omega)=|\frac{3}{2j+4}|=\frac{3}{\sqrt{16+4}}=\frac{3}{2\sqrt{5}}=\frac{3\sqrt{5}}{10}$\\
$A_y=3\cdot A(\omega)=\frac{9\sqrt{5}}{10}$\\
$\varphi_y=\arg G(j\omega)+\varphi_u$\\
$G(j\omega)=\frac{3}{2j+4}=\frac{3(4-2j)}{20}=\frac 3 5 -\frac{3}{10}j$\\
\fbox{\parbox{.24\linewidth}{
argument liczby $a+bi$ :\\
$\varphi=\begin{cases} \text{arctg}(\frac{b}{a}), a>0 \\ \text{arctg}(\frac ba)+\pi,a<0\end{cases}$
}}\\\\
$\arg G(j\omega)=\text{arctg}(-\frac{3}{10} \cdot \frac{5}{3})=\boxed{\text{arctg}(-\frac{1}{2})}$\\

%###################                 4.8.2              #################################%
\pagebreak
\subsection*{Zadanie 4.8.2} {\color{darkgray}
	Rozwiązanie równania różniczkowego\\
	$\dot{x}(t)=-x(t)+10\sin(5t+\frac\pi 3)$\\
	gdzie $x(0)=0, t \geqslant 0$ ma postać\\
	$x(t)=ae^{-4t}+A\sin(5t+\varphi)$\\
	Obliczyć $A$ i $\varphi$.\\
}\lineh
\\\\

%###################                 4.9.1              #################################%
\pagebreak
\subsection*{Zadanie 4.9.1} {\color{darkgray}
	Rozwiązanie równania różniczkowego\\
	$\ddot{x}(t)+\dot{x}(t)=-4x(t)+3\sin({\omega t})$\\
	gdzie $x(0)=0$, ($\dot{x}(0)$ - w domysle), $t \geq 0$ ma postać\\
	$x(t)=f(t)+A\sin({\omega t+\varphi})$\\
	znaleźć takie $\omega$, dla którego $A$ jest największe\\
}\lineh
\\\\
$\ddot{x}(t)+\dot{x}(t)+4x(t)=3\sin({\omega t})$\\
$u(t)=\sin(\omega t)=\frac{\ddot{x}(t)+\dot{x}(t)+4x(t)}{3}$\\
$Y(s)=G(s)\cdot U(s) \Rightarrow G(s)=\frac{Y(s)}{U(s)}$\\
Transformata Laplace'a\\
$X(s)=Y(s)$\\
\fbox{\parbox{.3\linewidth}{
	$\mathcal{L}\{f'\}=s\mathcal{L}\{f\}-f(0^+)$\\
	$\mathcal{L}\{f''\}=s^2\mathcal{L}\{f\}-sf(0^+)-f'(0^+)$
}}\\\\
$U(s)=\frac{s^2X(s)-s\cdot x(0)-\dot{x}(0)+sX(s)-x(0)+4X(s)}{3}=\frac{X(s)(s^2+s+4)}{3}$\\
$G(s)=\frac{X(s)}{\frac{X(s)(s^2+s+4)}{3}}=\frac{3 \cancel{X(s)}}{ \cancel{X(s)}(s^2+s+4)}=\frac{3}{s^2+s+4}$\\
$A=\underbrace{A_y}_{\text{wyjście}}=\underbrace{A_u}_{\text{wejście}} \cdot A(\omega)$\\
$A(\omega)=|G(\omega j)|$\\
$A$ będzie maksymalne dla $|G(\omega j)|$ maksymalnego\\
$|G(\omega j)| = |\frac{3}{-\omega^2+j\omega+4}|=|\frac{3}{\sqrt{(4-\omega^2)^2+\omega^2}}|=|\frac{3}{\sqrt{16-8\omega^2+\omega^4+\omega^2}}|=|\frac{3}{\sqrt{\omega^4-7\omega^2+16}}|$\\
$\frac{3}{\sqrt{\omega^4-7\omega^2+16}}$ maksymalne $\Leftrightarrow \sqrt{\omega^4-7\omega^2+16}$ minimalne\\
$z=\omega^2$\\
szukamy min funkcji $z^2-7z+16$\\
dodatni znak przy $z^2$, więc minimum będzie na wierzchołku, czyli $x_w=\frac{-b}{2a}=\frac 72$\\
$\omega=\sqrt\frac 72 \ \ \ \vee \ \ \ \omega=-\sqrt\frac 72$\\
$\omega=-\sqrt\frac 72$ odpada, bo nie może być $<0$\\
$\boxed{\omega=\sqrt\frac 72}$\\





%###################                 4.9.2              #################################%
\pagebreak
\subsection*{Zadanie 4.9.2} {\color{darkgray}
	Rozwiązanie równania różniczkowego\\
	$\ddot{x}(t)+\dot{x}(t)=-x(t)+12\sin({\omega t})$\\
	gdzie $x(0)=0$, ($\dot{x}(0)$ - w domysle), $t \geq 0$ ma postać\\
	$x(t)=f(t)+A\sin({\omega t})$\\
	znaleźć takie $\omega$, dla którego $A$ jest największe\\
}\lineh
\\\\
$u(t)=\sin({\omega t})$\\
$y(t)=x(t)$\\\\
$\begin{cases} \ddot{x}(t)+\dot{x}(t)=-x(t)+12u(t)\\y(t)=x(t)\end{cases}$\\\\
$\left.\begin{cases} u(t)=\frac{\ddot{x}(t)+\dot{x}(t)+x(t)}{12}\\y(t)=x(t)\end{cases}\right|\mathscr{L}$\\\\
$\begin{cases} U(s)=\frac{s^2\cdot X(s)-s\cdot x(0)-\dot{x}(0)+sX(s)-x(0)+X(s)}{12}\\Y(s)=X(s)\end{cases}$\\\\
$\begin{cases} U(s)=\frac{s^2\cdot X(s)+sX(s)+X(s)}{12}\\Y(s)=X(s)\end{cases}$\\\\\\
$G(S)=\frac{Y(s)}{U(s)}=\frac{12\cdot \cancel{X(s)}}{\cancel{X(s)}\cdot (s^2+s+1)}=\frac{12}{s^2+s+1}$\\\\
$G(j\omega)=\frac{12}{-\omega^2+j\omega+1}=-\frac{12\cdot \omega^2-12}{\omega^4-\omega^2+1}-j\frac{12\cdot \omega}{\omega^4-\omega^2+1}$
$A_y=A_u\cdot ku(\omega)$\\
$ku(\omega)=|G(j\omega)|=\sqrt{\left(\frac{12\cdot \omega^2-12}{\omega^4-\omega^2+1}\right)^2+\left(\frac{12\cdot \omega}{\omega^4-\omega^2+1}\right)^2}=12\cdot \sqrt{\frac{1}{\omega^4-\omega^2+1}}$\\\\\\
$A_y$ będzie max., gdy $ku(\omega)$ będzie max., tj. $\sqrt{\omega^4-\omega^2+1}$ będzie min.
$\omega \geq 0$, $min(\sqrt{\omega^4-\omega^2+1})$ dla $\omega=\frac{1}{\sqrt{2}}$


%###################                 4.9.3              #################################%
\pagebreak
\subsection*{Zadanie 4.9.3} {\color{darkgray}
	Rozwiązanie równania różniczkowego\\
	$\ddot{x}(t)+\dot{x}(t)=-5x(t)+15\sin({\omega t})$\\
	gdzie $x(0)=0$, ($\dot{x}(0)$ - w domysle), $t \geq 0$ ma postać\\
	$x(t)=f(t)+A\sin({\omega t+\varphi})$\\
	znaleźć takie $\omega$, dla którego $A$ jest największe\\
}\lineh
\\\\

%###################                 4.10.1              #################################%
\pagebreak
\subsection*{Zadanie 4.10.1} {\color{darkgray}
	Korzystając z kryterium Michajłowa zbadać stabilność asymptotyczną układu opisanego transmitancją $G(s)$:\\
	$G(s)=\frac{s^2+3s+1}{s^3+s^2-4s+9}$\\
}\lineh
\\\\
\fbox{\parbox{.6\linewidth}{
	\textbf{Kryterium Michajłowa}\\
	$G(s)=\frac{L(s)}{M(s)}$\\
	$M(s)=a_ns^n+a_{n-1}s^{n-1}+...+a_1s+a_0$\\
	Układ jest asymptotycznie stabilny jeśli przyrost argumentu $M(j\omega)$ rzędu $n$ przy zmianie $\omega$ od $-\infty$ do $+\infty$ wynosi $n\pi$: $\Delta Arg \ M(j\omega)|^{+\infty}_{-\infty}=n\pi$\\
}}\\\\
Ponieważ funkcja $M(j\omega)$ jest symetryczna względem osi rzeczywistej, więc wystarczy $\Delta Arg \ M(j\omega)|^{+\infty}_0=n\frac\pi 2$\\
inna postać kryterium (wynikająca z powyższego):
wystarczy pokazać, że charakterystyka częstotliwościowa funkcji $M(j\omega) \ \ 0<\omega <\infty$ przechodzi przez $n$ ćwiartek w kierunku dodatnim.\\
$n=3$ więc musi przechodzić przez I, II i III ćwiartkę\\
$M(s)=s^3+s^2-4s+9$\\
$M(j\omega)=-j\omega^3-\omega^2-4j\omega+9=\underbrace{9-\omega^2}_{Re}+j\underbrace{(-\omega^3-4\omega)}_{Im}$\\
$\omega=0 \Rightarrow M(j0)=9$\\
\begin{figure}[!h]
\begin{tikzpicture}
	\draw [color=red, thick](2,0) arc (0:90:2 and 1.5);
	\draw [color=red, thick](0,1.5) arc (90:180:1 and 1.5);
	\draw [color=red, thick](-1,0) arc (180:220:1.5 and 3);

	\draw[thick][->](-3,0)--(3,0) node [right=3pt]{$Re$};
	\draw[thick][->](0,-3)--(0,3) node [right=3pt]{$Im$};

	\draw (1.3,1.2) -- (2,1.2);
	\draw (1.3,1.2) -- (1.5,1.3);
	\draw (1.3,1.2) -- (1.5,1.1);

	\node at(4.2,1.2){powinny być tak, żeby był a.s.};


	\draw (-0.1,-2) -- (0.1,-2) node [left=3pt]{{-39}};
	\draw (2,-0.1) -- (2,0.1) node [below=4pt]{{9}};
\end{tikzpicture}
\end{figure}
\noindent\\
spr. gdzie przecina $Im$ dla $Re=0$\\
$9-\omega^2=0\Rightarrow\omega=3$\\
$Im=-3^3-4\cdot 3=-39$ {\color{lightgray} ($Im$ powinno być dodatnie, aby układ był asymptotycznie stabilny)}\\
spr. czy przecina oś $Re$ w przedziale $(-\infty, 0)$\\
$Im=0\Rightarrow -\omega(\omega^2+4)=0$\\
$\omega=0 \ \ \ \omega^2=-4$\\
$Re=9+4=13$\\
więc, układ nie jest asymptotycznie stabilny.

%###################                 4.10.2              #################################%
\pagebreak
\subsection*{Zadanie 4.10.2} {\color{darkgray}
	Korzystając z kryterium Michajłowa zbadać stabilność asymptotyczną układu opisanego transmitancją $G(s)$:\\
	$G(s)=\frac{s^2+s+1}{s^3-s^2+2s-2}$\\
}\lineh
\\\\

%  \left[\begin{array}{cc}\end{array}\right]

\pagebreak
\section*{Tydzień 5}
Wprowadzenie do układów nieliniowych - Linearyzacja
%###################                 5.1.1              #################################%
\subsection*{Zadanie 5.1.1} {\color{darkgray}
	Dany jest system dynamiczny\\
	$\dot{x}(t)= \cos(x(t))e^{-x(t)^2}$\\
	Wyznaczyć jego punkty równowagi i za pomocą I metody Lapunowa zbadać ich stabilność.\\
}\lineh
\\\\
$\dot{x}(t)=\cos(x(t))e^{-x(t)^2}$\\
$\dot{x}(t)=f(x(t))$\\
$x_r$ jest punktem równowagi $ \Leftrightarrow f(x_r)=0$\\
$f(x_r)=\cos(x_r)\cdot \underbrace{e^-x_r^2}_{<0}=0 \Rightarrow cos(x_r)=0 \Rightarrow x_r=\frac{\pi}{2}+k\pi, k \in \mathbb{Z}$\\
$\boxed{\begin{aligned}
\text{System zlinearyzowany: } \dot{x}(t)=J(x_r)x(t)\\
J(x)=\left[ \begin{array}{ccc} 
 \frac{\partial f_1}{\partial x_1}(x) & \cdots &  \frac{\partial f_1}{\partial x_n}(x)\\
\vdots & \ddots & \vdots\\
\frac{\partial f_n}{\partial x_1}(x) &\cdots & \frac{\partial f_n}{\partial x_n}(x) 
  \end{array}\right] f(x)= \left[ \begin{array}{c}  f_1(x)   \\ \vdots \\ f_n(x)    \end{array}\right]
\end{aligned}}$\\
$J(x)=\frac{\partial f}{\partial x} = - \sin(x) \cdot e^{-x^2}+\cos(x) \cdot (-2xe^{-x^2})=-e^{-x^2}(\sin(x)+2x\cos(x))$\\
$J(x_r)=\underbrace{-e^{-(\frac{\pi}{2}+k\pi)^2}}_{<0}(\underbrace{\sin(\frac{\pi}{2}+k\pi)}_{=1 \vee =-1})+\underbrace{2(\frac{\pi}{2}+k\pi)\cos(\frac{\pi}{2}+k\pi)}_{=0}=-e^{-(\frac{\pi}{2}+k\pi)^2} (\sin(\frac{\pi}{2}+k\pi))$\\
\\
\fbox{\parbox{.5\linewidth}{
\textbf{I metoda Lapunowa} \\
Punkt równowagi, dla którego system zlinearyzowany jest asymptotycznie stabilny jest lokalnie asymptotycznie stabilny. Jeżeli zaś chociaż jedna z wartości własnych macierzy systemu zlinearyzowanego ma dodatnią część rzeczywistą to punkt równowagi jest niestabilny.
}}\\
$\lambda = -e^{-(\frac{\pi}{2}+k\pi)^2}\cdot \sin(\frac{\pi}{2}+k\pi)$\\
\textbf{niestabilny :}\\
$ -e^{-(\frac{\pi}{2}+k\pi)^2}\cdot \sin(\frac{\pi}{2}+k\pi)>0 \Rightarrow \sin(\frac{\pi}{2}+k\pi)=-1 \Rightarrow x_r=\frac{\pi}{2}+(2k\pi+1)\pi,\ \  k \in \mathbb{Z}$\\
(z Hurwitza)\\
$-e^{-(\frac{\pi}{2}+k\pi)^2}\cdot \sin(\frac{\pi}{2}+k\pi)>0 \Rightarrow \sin(\frac{\pi}{2}+k\pi)= 1 \Rightarrow x_r=\frac{\pi}{2}+2k\pi,\ \  k \in \mathbb{Z}$\\
$ \left[ \begin{array}{cc}    1&0 \\0&   e^{-(\frac{\pi}{2}+k\pi)^2}\cdot \sin(\frac{\pi}{2}+k\pi) \end{array}\right]$

\pagebreak
%###################                 5.1.2              #################################%
\subsection*{Zadanie 5.1.2} {\color{darkgray}
	Dany jest system dynamiczny\\
	$\dot{x}_1(t)=x_2(t)$\\
	$\dot{x}_2(t)=-2x_1(t)-3x_1(t)^2-x_2(t)$\\
	Wyznaczyć jego punkty równowagi i za pomocą I metody Lapunowa zbadać ich stabilność.\\
}\lineh
\\\\
$J(x)=\left[ \begin{array}{cc}  0&1\\-2-6x_1 & -1   \end{array}\right]$\\
$\begin{cases}x_2=0\\-2x_1-3x_1^2-x_2=0\end{cases}$\\
$-2x_1-3x_1^2=0$\\
$x_1(2+3x_1)=0$\\
$x_1=0\ \  \vee \ \ x_1=-\frac23$\\
$\begin{array}{lll}
x_r= \left[ \begin{array}{c}   0\\0    \end{array}\right] &\ \ \ \vee \ \ \ & x_r= \left[ \begin{array}{c}   -\frac 23\\0    \end{array}\right] \\
J(x_r)=\left[ \begin{array}{cc}  0&1\\-2&-1    \end{array}\right] && J(x_r)=\left[ \begin{array}{cc}   0&1\\2&-1    \end{array}\right]\\
\left| \begin{array}{cc}  -\lambda & 1 \\-2&-1-\lambda    \end{array}\right|&&\left| \begin{array}{cc}  -\lambda & 1 \\2&-1-\lambda    \end{array}\right|\\
=(-\lambda)(-1-\lambda)+2=\lambda^2+\lambda+2&&=(-\lambda)(-1-\lambda)-2=\lambda^2+\lambda-2\\
\Delta=-7&&\Delta=9\\
\lambda=-\frac 12 \pm \frac{\sqrt 7}{2}i&&\lambda=\frac{-1 \pm 3}{2}\\
\lambda = -\frac 12 >0 \Rightarrow \text{Stabilny} && \lambda = 1 \ \ \vee \ \ \lambda=-2\\
 && \lambda = 1 >0 \Rightarrow \text{Niestabilny}
\end{array}$\\

%\begin{array}{c}\text{\circled{2}}\\x_r\end{array}

\pagebreak
%###################                 5.2.1              #################################%
\subsection*{Zadanie 5.2.1} {\color{darkgray}
	Wyznaczyć punkty równowagi układu generatora synchronicznego, który jest systemem dynamicznym opisanym następującymi równaniami\\
	$\dot{x}_1=x_2$\\
	$\dot{x}_2=-Dx_2-\sin x_1 + \sin \delta_0$\\
}\lineh
\\\\
$\begin{cases}\dot{x}_1=x_2 \\ \dot{x}_2=-Dx_2-\sin x_1 + \sin \delta_0 \end{cases}$\\
$f(x)= \left[ \begin{array}{c}   x_2  \\  -Dx_2-\sin x_1 + \sin \delta_0  \end{array}\right]$\\
$\left[ \begin{array}{c}   x_2  \\  -Dx_2-\sin x_1 + \sin \delta_0  \end{array}\right] = \left[ \begin{array}{c}  0\\0  \end{array}\right]$\\
$x_2=0$\\
$-\sin x_1 + \sin \delta_0 = 0 \Rightarrow \sin\delta_0=\sin x_1 \Rightarrow$\\
$\Rightarrow x_1=\delta_0+2k\pi \ \ \vee\ \  x_1=-\delta_0+(2k+1)\pi, \ \ \ k \in \mathbb{Z}$\\




\pagebreak
%###################                 5.2.2              #################################%
\subsection*{Zadanie 5.2.2} {\color{darkgray}
	Wyznaczyć punkty równowagi dla obwodu Chuy, który jest systemem dynamicznym opisanym następującymi równaniami:\\
	$C_1\dot{x}_1(t)=-\frac 1R x_1(t)+\frac 1R x_2(t)-g(x_1(t))$\\
	$C_2\dot{x}_2(t)= \frac 1R x_1(t) -\frac 1R x_2(t)+x_3(t)$\\
	$L\dot{x}_3(t)=-x_2(t)-R_0x_3(t)$\\
	przy czym $g(v)=g_1v+g_2v^3$\\
}\lineh
\\\\
$\begin{cases} 
-\frac{1}{RC_1}x_1+\frac{1}{RC_1}x_2-\frac{g_1}{C_1}x_1-\frac{g_2}{C_1}x_1^3=0\\
\frac{1}{RC_2}x_1 -\frac{1}{RC_2}x_2+\frac{x_3}{C_2}=0\\
-\frac 1Lx_2 -\frac{R_0}{L}x_3=0 \ \ \ \ \ \ \ \ \ \ \ \Rightarrow x_2=-R_0x_3
\end{cases}$\\
$\begin{cases} 
g_1Rx_1+g_2Rx_1^3+x_1=x_2\\
x_1+Rx_3=x_2\\
x_2=-R_0x_3
\end{cases}$\\
z drugiego i trzeciego:\\
$x_3=\frac{-x_1}{R+R_0}$\\
z pierwszego i drugiego:\\
$g_1Rx_1+g_2Rx_1^3=Rx_3$\\
$g_1x_1+g_2x_1^3=x_3$\\
podstawiam $x_3$ z trzeciego:\\
$g_1x_1+g_2x_1^3=\frac{-x_1}{R+R_0}$\\
$g_1x_1+x_1^3g_2+\frac{x_1}{R+R_0}=0$\\
$x_1^3g_2+x_1(g_1+\frac{1}{R+R_0})=0$\\
podstawiam pomocnicze zmienne:\\
$a = g_2, \ \ \ b=g_1+\frac{1}{R+R_0}$\\
$ax_1^3+bx_1=0$\\
$x_1(ax_1^2+b)=0$\\
$\begin{array}{lll}   
\circled{1} \ x_1=0 & \ \ \vee \ \ & ax_1^2+b=0\\
&&\circled{2} \ x_1=\sqrt{\frac{-b}{a}} \ \ \vee \ \  \circled{3} \ x_1=-\sqrt{\frac{-b}{a}}
 \end{array}$\\
podstawiam $x_1=0$:\\
$x_r=\left[ \begin{array}{c}   x_{1r}\\ x_{2r}\\x_{3r}   \end{array}\right]
=\left[ \begin{array}{c}   x_{1}\\ x_{2}\\x_{3}   \end{array}\right]$\\\\
$\circled{1} \ x_r=\left[ \begin{array}{c}   0\\0\\0   \end{array}\right]$\\\\\\
$\circled{2} \ x_r=\left[ \begin{array}{c}   
	\sqrt{\frac{-g_1-\frac{1}{R+R_0}}{g_2}}\\
\\
	-R_0\frac{-\sqrt{\frac{-g_1-\frac{1}{R+R_0}}{g_2}}}{R+R_0}   \\
\\
	\frac{-\sqrt{\frac{-g_1-\frac{1}{R+R_0}}{g_2}}}{R+R_0}
\end{array}\right]$\\\\\\
$\circled{3} \ x_r=\left[ \begin{array}{c}   
	-\sqrt{\frac{-g_1-\frac{1}{R+R_0}}{g_2}}\\
\\
	-R_0\frac{\sqrt{\frac{-g_1-\frac{1}{R+R_0}}{g_2}}}{R+R_0}   \\
\\
	\frac{\sqrt{\frac{-g_1-\frac{1}{R+R_0}}{g_2}}}{R+R_0}
\end{array}\right]$



\pagebreak
%###################                 5.3.1              #################################%
\subsection*{Zadanie 5.3.1} {\color{darkgray}
	Dla jakich wartości parametru $\epsilon$ zerowy punkt równowagi układu zwanego oscylatorem Van der Pola będzie niestabilny\\
	$\ddot{x}(t)-\epsilon(1-x(t)^2)\dot{x}(t)+x(t)=0$\\\\
}\lineh
\\\\
$\begin{cases} x_1=x \\ x_2 =\dot{x} \end{cases} \begin{cases} \dot{x}_1=\dot{x}=x_2 \\ \dot{x}_2=\ddot{x}= \epsilon(1-x(t)^2) \cdot \dot{x}(t)-x(t)=\epsilon(1-x_1^2) \cdot x_2 - x_1 \end{cases}$\\
$f(x)=\left[ \begin{array}{c}   x_2  \\  \epsilon(1-x_1^2) \cdot x_2 - x_1  \end{array}\right] = \left[ \begin{array}{c}  f_1(x)   \\ f_2(x)   \end{array}\right]$\\
$\begin{cases} x_2=0 \\ \epsilon(1-x_1^2) \cdot x_2 - x_1 = 0\end {cases} \Rightarrow \begin{cases}x_2=0 \\ x_1 = 0 \end {cases} \ \ \ x_r = \left[ \begin{array}{c}     0\\0   \end{array}\right]$\\
$J(x)=\left[ \begin{array}{cc}   0 &1  \\ -2\epsilon x_1 x_2 -1 & \epsilon(1-x_1^2)   \end{array}\right] \ \ \ \ \ \ \ 
{\color{lightgray}\boxed{\left[ \begin{array}{cc}    \frac{\partial f_1}{\partial x_1} &\frac{\partial f_1}{\partial x_2} \\ \frac{\partial f_2}{\partial x_1} & \frac{\partial f_2}{\partial x_2}   \end{array}\right]}}$\\
$J(x_r)= \left[ \begin{array}{cc}    0&1 \\-1 & \epsilon    \end{array}\right]$\\
Z Lapunowa:\\
$(- \lambda)(\epsilon - \lambda)+1 = \lambda^2 - \lambda \epsilon +1 = 0$\\
$\Delta = \epsilon^2-4 \Rightarrow \lambda =\frac{\epsilon \pm \sqrt{\epsilon^2-4}}{2}$\\
niestabilny: Jeżeli część rzeczywista $>0 \Rightarrow \frac{\epsilon}{2}>0 \Rightarrow \boxed{\epsilon >0}$\\
asymptotycznie stabilny : $\left[ \begin{array}{cc}    -\epsilon & 0 \\ 1 & 1   \end{array}\right]$ Hurwitz $-\epsilon>0 \Rightarrow \boxed{\epsilon<0}$\\



\pagebreak
%###################                 5.4.1              #################################%
\subsection*{Zadanie 5.4.1} {\color{darkgray}
	Dla jakich wartości parametru $a$ linearyzacja przestaje spełniać warunki twierdzenia Grobmana-Hartmana dla układu opisanego równaniami:\\
	$\dot{x_1}(t)=-x_2(t)+(a-x_1(t)^2-x_2(t)^2)x_1(t)$\\
	$\dot{x_2}(t)=x_1(t)+(a-x_1(t)^2-x_2(t)^2)x_2(t)$\\\\
}\lineh
\\\\
$\dot{x_1}(t)=-x_2(t)+(a-x_1(t)^2-x_2(t)^2)x_1(t)=f_1(x(t))$\\
$\dot{x_2}(t)=x_1(t)+(a-x_1(t)^2-x_2(t)^2)x_2(t)=f_2(x(t))$\\
$f(x)=\left[ \begin{array}{c}     f_1(x) \\ f_2(x)   \end{array}\right]$\\
$\begin{cases} -x_2+(a-x_1^2-x_2^2)x_1=0 \\ x_1+(a-x_1^2-x_2^2)x_2=0\end{cases}  $\\
Zauważamy, że albo $x_1=x_2=0$ albo dla $x_2 \neq 0 \wedge x_1 \neq 0$ :\\
$\begin{cases} -\frac{x_2}{x_1}+(a-x_1^2-x_2^2)=0 \\ \frac{x_1}{x_2}+(a-x_1^2-x_2^2)=0\end{cases}  \Rightarrow \frac{-x_2}{x_1} = \frac{x_1}{x_2} \Rightarrow -x_2^2=x_1^2 \Rightarrow x_1=x_2=0$ (sprzeczność)\\
więc $x_r=\left[ \begin{array}{c}     0\\0   \end{array}\right]$\\
$J(x)=\left[ \begin{array}{cc}   a-3x_1^2-x_2^2 & -1-2x_2x_1 \\ 1-2x_1x_2 & a-x_1^2-3x_2^2    \end{array}\right]$\\
$J(x_r)=\left[ \begin{array}{cc}    a & -1 \\ 1 & a    \end{array}\right]$\\
z tw. Grobmana-Hartmana:\\
$\begin{array}{ll}
\det(j\omega I-J(x_r)) \neq 0, \ \ \ \omega \in \mathbb{R} & J(x_r)\text{ nie ma wartości własnych na osi urojonej}\\
 \left| \begin{array}{cc}     j\omega-a& -1 \\ 1 & j\omega-a    \end{array}\right|=0 &  \left[ \begin{array}{cc}    a-\lambda & -1 \\ 1 & a- \lambda   \end{array}\right]\\
(j\omega-a)^2+1=0 & (a-\lambda)^2+1=0\\
j\omega-1= \pm j \Rightarrow \boxed{ a=0} & a^2-2a\lambda +\lambda^2+1 =0\\
&\lambda^2-2a\lambda+a^2+1 = 0\\
&\Delta=4a^2-4a^2-4\\
&\sqrt{\Delta}=2i\\
&\lambda=\frac{2a \pm 2i}{2} = a \pm i\\
&\text{dla } a=0 \text{ wartości własne są na osi urojonej}
\end{array}$\\


\pagebreak
%###################                 5.5.1              #################################%
\subsection*{Zadanie 5.5.1} {\color{darkgray}
	Dla jakich wartości parametru $a$ zerowy punkt równowagi układu opisanego równaniami\\
	$\dot{x_1}(t)=x_2(t)+(a-x_1(t)^2-x_2(t)^2)x_1(t)$\\
	$\dot{x_2}(t)=x_1(t)+(a-x_1(t)^2-x_2(t)^2)x_2(t)$\\
	będzie niestabilny.\\
}\lineh
\\\\
$x_r=\left[ \begin{array}{c}     0\\0   \end{array}\right]$\\
$J(x)=\left[ \begin{array}{cc}    a-3x_1^2 -x_2^2&  1-2x_2x_1 \\ 1-2x_1x_2 & a-x_1^2-3x_2^2  \end{array}\right]$\\
$J(x_r)=\left[ \begin{array}{cc}    a&1\\1&a  \end{array}\right]$\\
$\left| \begin{array}{cc}    a-\lambda&1\\1&a-\lambda  \end{array}\right| =0$\\
$(a-\lambda)^2-1=0$\\
$(a-\lambda-1)(a-\lambda+1)=0$\\
$\lambda=a-1 \ \ \ \vee \ \ \ \lambda=a+1$\\
niestabilny, gdy $Re(\lambda)>0$\\
$\begin{array}{lll}     a-1>0& \ \ \  &a+1>0  \\ a>1 && a>-1 \end{array}$\\
$(a>1 \ \ \vee \ \ a>-1 ) \Rightarrow \boxed{a>-1}$\\
\\
{\color{red} \textbf{ Alternatywnie: Bez podanego punktu równowagi}}\\
$\dot{x_1}(t)=x_2(t)+(a-x_1(t)^2-x_2(t)^2)x_1(t)=f_1(x(t))$\\
$\dot{x_2}(t)=x_1(t)+(a-x_1(t)^2-x_2(t)^2)x_2(t)=f_2(x(t))$\\
$f(x)= \left[ \begin{array}{cc}    f_1(x)\\f_2(x)    \end{array}\right]$\\
$\begin{cases}x_2+(a-x_1^2-x_2^2)x_1=0 \\ x_1+(a-x_1^2-x_2^2)x_2=0\end{cases}$\\
Zauważamy, że albo $x_1=x_2=0$ albo dla $x_1\neq 0 \wedge x_2 \neq 0 $ :\\
$\begin{cases} \frac{x_2}{x_1}+(a-x_1^2-x_2^2)=0 \\ \frac{x_1}{x_2}+(a-x_1^2-x_2^2)=0\end{cases} 
\Rightarrow \frac{x_2}{x_1}=\frac{x_1}{x_2} \Rightarrow 
\begin{array}{c}x_2^2 = x_1^2 \\x_1=x_2 \vee x_1=-x_2 \end{array}$\\
więc $\begin{array}{c}\text{\circled{1}}\\x_r\end{array}=\left[ \begin{array}{c} k\\k\end{array}\right] \vee
\begin{array}{c}\text{\circled{2}}\\x_r\end{array}=\left[ \begin{array}{c} k\\-k\end{array}\right],  \ \ \ k \in \mathbb{R}$\\
$J(x)=\left[ \begin{array}{cc}    a-3x_1^2 -x_2^2&  1-2x_2x_1 \\ 1-2x_1x_2 & a-x_1^2-3x_2^2  \end{array}\right]$\\
$\begin{array}{lll}
J( \begin{array}{c}\text{\circled{1}}\\x_r\end{array})= \left[ \begin{array}{cc}  a-4k^2 & 1-2k^2 \\1-2k^2 & a-4k^2 \end{array}\right] &\vee&
J( \begin{array}{c}\text{\circled{2}}\\x_r\end{array})= \left[ \begin{array}{cc}  a-4k^2 & 1+2k^2 \\1+2k^2 & a-4k^2 \end{array}\right]\\
(a-4k^2-\lambda)^2-(1-2k^2)^2=0         &\vee&       (a-4k^2-\lambda)^2-(1+2k^2)^2=0 \\
(a-4k^2-\lambda-1+2k^2)(a-4k^2-\lambda+1-2k^2)=0 &\vee& (a-4k^2-\lambda-1-2k^2)(a-4k^2-\lambda+1+2k^2)=0 \\
\lambda = a-1-2k^2 \vee \lambda=a+1-6k^2 &\vee& \lambda = a-1-6k^2 \vee \lambda=a+1-2k^2 \\
niestebilny:&&\\
Re \lambda >0 &&\\
a-1-2k^2>0 \vee a+1-6k^2>0 &&a-1-6k^2>0 \vee a+1-2k^2>0\\
\begin{array}{c}\text{\circled{1}}\\a>1+2k^2\end{array}
\begin{array}{c}\text{\circled{2}}\\a>6k^2-1\end{array} && 
\begin{array}{c}\text{\circled{3}}\\a>1+6k^2\end{array}
\begin{array}{c}\text{\circled{4}}\\a>2k^2-1\end{array}
\end{array}$\\
odp. niestabilny dla $a>\circled{1} \vee a>\circled{2} \vee a>\circled{3} \vee a>\circled{4} $\\



\pagebreak
%###################                 5.5.2              #################################%
\subsection*{Zadanie 5.5.2} {\color{darkgray}
	Dla jakich wartości parametru $a$ zerowy punkt równowagi układu opisanego równaniami\\
	$\dot{x_1}(t)=-x_2(t)+(a-x_1(t)^2-x_2(t)^2)x_1(t)$\\
	$\dot{x_2}(t)=x_1(t)+(a-x_1(t)^2-x_2(t)^2)x_2(t)$\\
	będzie niestabilny.\\
}\lineh
\\\\
$x_r=\left[ \begin{array}{c}     0\\0   \end{array}\right]$\\
$J(x)=\left[ \begin{array}{cc}   a-3x_1^2-x_2^2 & -1-2x_2x_1 \\ 1-2x_1x_2 & a-x_1^2-3x_2^2    \end{array}\right]$\\
$J(x_r)=\left[ \begin{array}{cc}    a & -1 \\ 1 & a    \end{array}\right]$\\
$\left| \begin{array}{cc}     a-\lambda & -1 \\ 1 & a-\lambda   \end{array}\right|=0$\\
$\lambda^2-2a\lambda+a^2+1=0$\\
$\Delta=-4$\\
$\lambda = \frac{2a \pm 2i}{2}=a \pm i$\\
niestabilny, gdy $Re(\lambda)>0$\\
$\boxed{a>0}$\\



\pagebreak
%###################                 5.6.1              #################################%
\subsection*{Zadanie 5.6.1} {\color{darkgray}
	Dla jakich wartości parametru $a$ zerowy punkt równowagi układu opisanego równaniami\\
	$\dot{x_1}(t)=-x_2(t)+(a-x_1(t)^2-x_2(t)^2)x_1(t)$\\
	$\dot{x_2}(t)=x_1(t)+(a-x_1(t)^2-x_2(t)^2)x_2(t)$\\
	będzie asymptotycznie stabilny.\\
}\lineh
\\\\
$x_r=\left[ \begin{array}{c}     0\\0   \end{array}\right]$\\
$J(x)=\left[ \begin{array}{cc}   a-3x_1^2-x_2^2 & -1-2x_2x_1 \\ 1-2x_1x_2 & a-x_1^2-3x_2^2    \end{array}\right]$\\
$J(x_r)=\left[ \begin{array}{cc}    a & -1 \\ 1 & a    \end{array}\right]$\\
$\left| \begin{array}{cc}     a-\lambda & -1 \\ 1 & a-\lambda   \end{array}\right|=0$\\
$\lambda^2-2a\lambda+a^2+1=0$\\
\fbox{\parbox{.5\linewidth}{
\textbf{Macierz Hurwitz'a} \\
Wielomian charakterystyczny: $a_0\lambda^n+a_1\lambda^{n-1}+a_{n-1}\lambda+a_n$\\
$\left[ \begin{array}{cccccc}  
 a_1&a_3&a_5&a_7&\cdots&0\\
 a_0&a_2&a_4&a_6&\cdots&0\\
     0&a_1&a_3&a_5&\cdots&0\\
     0&a_0&a_2&a_4&\cdots&0\\
     0&    0&a_1&a_3&\cdots&0\\
\vdots&\vdots&\vdots&\vdots&\ddots&\vdots\\
0&0&0&0&\cdots&a_n
 \end{array}\right]$\\
}}\\\\\\
$\left[ \begin{array}{cc}   -2a & 0 \\ 1 & a^2+1    \end{array}\right]$\\\\\\
\fbox{\parbox{.5\linewidth}{
\textbf{Kryterium Hurwitz'a} \\
Jeśli wszystkie minory wiodące są większe od zera, to jest asymptotycznie stabilny.
}}\\\\
$-2a>0 \Rightarrow a<0$\\
$-2a(a^2+1)>0 \Rightarrow \boxed{a<0}$\\

%  \left[ \begin{array}{c}     \\    \end{array}\right] 

\pagebreak
\section*{Autorzy:}

\subsection*{Skład:} 
Jacek Pietras\\
Grzegorz Tokarz

\subsection*{Rozwiązania:} 
Magdalena Warzesia\\
Ania Szarawara\\
irytek102\\
Gniewomir\\
Jacek Pietras\\
Grzegorz Tokarz

\subsection*{Komentarze:} 
\end{document}


%  \left[ \begin{array}{cc}     &\\&    \end{array}\right]
%  \left[ \begin{array}{c}     \\    \end{array}\right]



