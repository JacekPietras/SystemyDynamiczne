
\pagebreak
\section*{Tydzień 4}
Analiza częstotliwościowa systemów dynamicznych
\subsection*{Zadanie 4.1.2} {\color{darkgray}
Układ jest opisany równaniami stanu w postaci\\\\
$\begin{cases} \dot{x}(t)=Ax(t)+Bu(t)\\y(t)=Cx(t)\end{cases}$\\\\
z macierzami\\\\
$A=\left[\begin{array}{cc}2&0\\-5&0\end{array}\right],
B=\left[\begin{array}{c}2\\2\end{array}\right],
C=\left[\begin{array}{cc}
0&1
\end{array}\right]
$\\\\
Znaleźć transmitancję operatorową tego układu przy założeniu zerowych warunków początkowych $x(0)=0$\\\\
Rozwiązanie:\\
$G(s)=C(sI-A)^{-1}B$\\
$G(s)=\left[\begin{array}{cc}0&1\end{array}\right]
\cdot
\left[\begin{array}{cc}{s-2}&0\\5&s\end{array}\right]^{-1}
\cdot
\left[\begin{array}{c}2\\2\end{array}\right]=\left[\begin{array}{cc}0&1\end{array}\right]
\cdot 
\left[\begin{array}{cc}{\frac{1}{s-2}}&0\\-5&\frac{1}{s}\end{array}\right]
\cdot
\left[\begin{array}{c}2\\2\end{array}\right]=\frac{2}{s}-\frac{10}{s\cdot(s-2)}=\frac{2s-14}{s\dot{(s-2)}}
$
\pagebreak
%------------------------------------------------------------------------------------------------------------
\subsection*{Zadanie 4.6.2} {\color{darkgray}
\begin{figure}[!h]
\begin{tikzpicture}
\draw[scale=0.8, transform shape]
(0,0) to[american current source,l=$u(t)$] 
(0,3) to[C,l=C,i=$i_c$] (3,3) to [european resistor,l=R] (3,0) -- (0,0);
\draw[->] (3,0.5) -- (3,2) node[right,pos=.5] {$x_1=U_R$};
\end{tikzpicture}
\end{figure}
\noindent Przeanalizować układ z rysunku i znaleźć równania opisujące ten układ. Za wyjście
przyjąć napięcie na oporniku. Znaleźć transmitancję operatorową i widmową układu\\
Zakladamy ze: $R=1000\Omega,C=1mF$\\\\
Rozwiązanie:\\
\noindent $\begin{cases}
u(t)-u_c(t)-u_R(t)=0\\
y(t)=u_R(t)
\end{cases}$\\\\
$\begin{cases}
u(t)-u_c(t)-RC\dot{u_c}(t)=0\\
y(t)=RC\dot{u_c}(t)
\end{cases}$\\\\
$\left.\begin{cases}
u(t)=+u_c(t)+RC\dot{u_c}(t)\\
y(t)=RC\dot{u_c}(t)
\end{cases}\right|\mathscr{L}$\\
$\begin{cases}
U(s)=U_c(s)+sRC\cdot U_c(s)+u_c(0)\\
Y(s)=sRC\cdot U_c(s)+u_c(0)
\end{cases}$\\
$G(s)=\frac{Y(s)}{U(s)}=\frac{sRC\cdot U_c(s)+u_c(0)}{U_c(s)+sRC\cdot U_c(s)+u_c(0)}=\frac{sRC\cdot U_c(s)}{U_c(s)+sRC\cdot U_c(s)}=\frac{\cancel{U_c(s)}\cdot sRC}{\cancel{U_c(s)}\cdot (sRC+1)}$\\\\
Podstawiamy R i C\\
$G(s)=\frac{s}{s+1}$\\
$G(j\omega)=\frac{j\omega}{j\omega+1}=\frac{j\omega}{j\omega+1}\cdot \frac{1-j\omega}{1-j\omega}=\frac{j\omega+\omega^2}{1+\omega^2}=\frac{\omega^2}{\omega^2+1}+j\frac{\omega}{\omega^2+1}$
%------------------------------------------------------------------------------------------------------------
\pagebreak
\subsection*{Zadanie 4.9.2} {\color{darkgray}
Rozwiązanie równania różniczkowego\\\\
$\ddot{x}(t)+\dot{x}(t)=-x(t)+12\sin({\omega t})$\\
gdzie $x(0)=0$, ($\dot{x}(0)$ - w domysle), $t \geq 0$ ma postać\\\\
$x(t)=f(t)+A\sin({\omega t})$\\
znalezc takie $\omega$, dla którego A jest największe\\\\
Rozwiązanie:\\
$u(t)=\sin({\omega t})$\\
$y(t)=x(t)$\\\\
$\begin{cases} \ddot{x}(t)+\dot{x}(t)=-x(t)+12u(t)\\y(t)=x(t)\end{cases}$\\\\
$\left.\begin{cases} u(t)=\frac{\ddot{x}(t)+\dot{x}(t)+x(t)}{12}\\y(t)=x(t)\end{cases}\right|\mathscr{L}$\\\\
$\begin{cases} U(s)=\frac{s^2\cdot X(s)-s\cdot x(0)-\dot{x}(0)+sX(s)-x(0)+X(s)}{12}\\Y(s)=X(s)\end{cases}$\\\\
$\begin{cases} U(s)=\frac{s^2\cdot X(s)+sX(s)+X(s)}{12}\\Y(s)=X(s)\end{cases}$\\\\\\
$G(S)=\frac{Y(s)}{U(s)}=\frac{12\cdot \cancel{X(s)}}{\cancel{X(s)}\cdot (s^2+s+1)}=\frac{12}{s^2+s+1}$\\\\
$G(j\omega)=\frac{12}{-\omega^2+j\omega+1}=-\frac{12\cdot \omega^2-12}{\omega^4-\omega^2+1}-j\frac{12\cdot \omega}{\omega^4-\omega^2+1}$
$A_y=A_u\cdot ku(\omega)$\\
$ku(\omega)=|G(j\omega)|=\sqrt{\left(\frac{12\cdot \omega^2-12}{\omega^4-\omega^2+1}\right)^2+\left(\frac{12\cdot \omega}{\omega^4-\omega^2+1}\right)^2}=12\cdot \sqrt{\frac{1}{\omega^4-\omega^2+1}}$\\\\\\
$A_y$ będzie max., gdy $ku(\omega)$ będzie max., tj. $\sqrt{\omega^4-\omega^2+1}$ będzie min.
$\omega \geq 0$, $min(\sqrt{\omega^4-\omega^2+1})$ dla $\omega=\frac{1}{\sqrt{2}}$
%------------------------------------------------------------------------------------------------------------
